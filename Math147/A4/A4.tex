\documentclass{article}
\usepackage{mathtools,amssymb,amsthm} % imports amsmath
\usepackage{enumitem}
\usepackage[a4paper, total={6in, 9in}]{geometry}
\begin{document}


\begin{enumerate}[leftmargin=*, label=\arabic*.]
  \item 
  For a), To prove that AB is bounded above, we take an arbitrary element $a \in A$ and $b \in B$. By the definition of a supremum, $sup(A) \geq a$ and $sup(B) \geq b$. This means that, especially considering $a, b \geq 0$: 
  \begin{align*}
      bsup(A) &\geq ba \\
      sup(A)sup(B) &\geq sup(A)b \\
      sup(A)sup(B) &\geq ba
  \end{align*}

  Since $sup(A)sup(B) \geq ab$ for any arbitrary $ab$, it is an upper  bound for $AB$. It also implies $sup(AB) \leq sup(A)sup(B)$. \\
  
  For the second part, we define $M$ where $M = max(sup(A), sup(B))$. Let $\epsilon$ be arbitrary and $\epsilon > 0$. First, we analyze the case where $\epsilon < M^2$. We find $a \in A$, $b \in B$ s.t. $sup(A) - (M - \sqrt{M^2 - \epsilon}) < a$ and $sup(B) - (M - \sqrt{M^2 - \epsilon}) < b$. We perform the same inequalities from earlier for $(sup(A) - (M - \sqrt{M^2 - \epsilon}))(sup(B) - (M - \sqrt{M^2 - \epsilon})) < ab$ and expand it. To make things less messy, $\hat{\epsilon} = M - \sqrt{M^2 - \epsilon}$
  \begin{align*}
      sup(A)sup(B)-2M\hat{\epsilon} + \epsilon^2 \leq sup(A)sup(B)-sup(A)\hat{\epsilon} - sup(B)\hat{\epsilon} + \epsilon^2 &< ab\\
      sup(A)sup(B) + ((\hat{\epsilon}-M)^2 - M^2) &< ab\\
      sup(A)sup(B) + (((M - \sqrt{M^2 - \epsilon})-M)^2-M^2) &< ab\\
      sup(A)sup(B) - \epsilon &< ab
  \end{align*}
  For the case $\epsilon \geq M^2$, select $a \in A, b \in B$ where $sep(A) - (M - \sqrt{M^2-\frac{M^2}{2}}) < a$ and $sep(B) - (M - \sqrt{M^2-\frac{M^2}{2}}) < b$. Then, $sup(A)sup(B) - \epsilon < sup(A)sup(B) - \frac{M^2}{2} < ab$. Since $\epsilon > 0$ was arbitrary, $sup(A)sup(B) \leq ab \leq sup(AB)$. Since $sup(A)sup(B) \leq sup(AB)$ and $sup(A)sup(B) \geq sup(AB)$, $sup(A)sup(B) = sup(AB)$ \\

  For b), let $A = [-5, -2]$ and $B = [-2, -1]$, so $sup(A) = -2$, $sup(B) = -1$, and $sup(A)sup(B) = 2$. However, $5 \in AB$, so $2$ is not an upper bound much less the least upper bound. Hence, $sup(A)sup(B) \neq sup(AB)$

  \item 
  For $(ii) \implies (i)$, we assume $(ii)$ so given an arbitrary $\epsilon > 0$ and its associated $\delta > 0$, we can find some arbitrary $x, y \in [a, b]$ where $-\delta < x - c, y - c < \delta$ or $ c - \delta < x, y < c + \delta$ and $|f(x) - f(c)| < \frac{\epsilon}{2}$ and $|f(c) - f(y)| < \frac{\epsilon}{2}$. Using the triangle inequality, it results in: 

  $$|f(x) - f(y)| \leq |f(x) - f(c)| + |f(c) - f(y)| < \epsilon$$

  $\epsilon$ is an upper bound for the set $X_\delta = \{|f(x) - f(y)|: x, y \in (c - \delta, c + \delta)\cap[a, b]\}$, so the $sup(X_\delta) \leq \epsilon$. Now, we denote the set W = $\{sup(X_\delta): \delta > 0\}$. Thus, given an arbitrarily $w \in W$, $w \geq 0$, so 0 is a lower bound for $W$. Since our choice for $\epsilon$ is arbitrary and there exist a $\delta > 0$ where $0 < \sup(X_\delta) \leq \epsilon$:
  
  $$inf\{sup\{|f(x) - f(y)|: x, y \in (c - \delta, c + \delta) \cap [a, b]\}: \delta > 0\} = 0$$

  For $(i) \implies (ii)$, we start with a proof by contrapositive. We assume there exist an $\epsilon > 0$ where there does not exist a $\delta > 0$, such that $|f(x) - f(c)| < \epsilon$ whenever $x \in [a, b]$ and $|x - c| < \delta$. Since we can set $\delta$ arbitrarily large, this means that $|f(x) - f(c)| \geq \epsilon$ for all $x \in [a, b]$. For any arbitrary $\delta > 0$ and an arbitrary $\hat{x} \in (c - \delta, c + \delta)$, considering $c \in (c - \delta, c + \delta)$: 
  
  $$sup\{|f(x)-f(y)|: x, y \in (c-\delta, c+\delta)\cap[a,b]\} \geq |f(\hat{x}) - f(c)| \geq \epsilon$$

  Thus $\epsilon$ is a lower bound for the set  $\{sup\{|f(x) - f(y)|: x, y \in (c - \delta, c + \delta) \cap [a, b]\}: \delta > 0\}$. Since $\epsilon > 0$, 0 is not the greatest lower bound. 
  

\end{enumerate}

\end{document}
