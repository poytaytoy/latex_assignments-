\documentclass{article}
\usepackage{enumitem}
\usepackage{mathtools,amssymb,amsthm} % imports amsmath
\usepackage{enumitem}
\usepackage[a4paper, total={6in, 9in}]{geometry}
\newtheorem{remark}{Remark}

\begin{document}

\begin{remark}
Let $a, b, c \in \mathbb{R}$ and $c > 0$ be in the form $|a-b| < c$. Expanding the absolute values gives $-c < a - b < c$. Thus, it follows that $b - c < a < b + c$.

\end{remark}

\begin{enumerate}

  \item

    \begin{enumerate}[label=\alph*)]

      \item 
      By contradiction, we assume that $b \neq a$. Thus, $|a - b| > 0$ and let $\epsilon = \frac{|a-b|}{4}$. There exist $N_1, N_2 \in \mathbb{R}$ and $n_1, n_2 \in \mathbb{N}$ where $|a_{n_1} - a| < \epsilon$ $\forall n_1 \geq N_1$ and $|a_{n_2} - b| < \epsilon$ $\forall n_2 \geq N_2$. We then denote $N = max\{N_1, N_2\}$. Thus, since $N_1 \leq N$ and $N_2 \leq N$, it implies that $|a_{n_1} - a| < \epsilon$ $\forall n_1 \geq P$ and $|a_{n_2} - b| < \epsilon$ $\forall n_2 \geq N$. \\

      Since $n_1$ and $n_2$ are both bounded below by $N$, for all $n \in \mathbb{N}, n \geq N$, it follows that $|a_n - a| < \epsilon$ and $|a_n - b| < \epsilon$ thus, by applying the triangle inequality: 
      
      $$|a -b| \leq |a - a_n| + |a_n - b| < 2\epsilon = 2 \cdot \frac{|a-b|}{4}$$
      $$|a - b| < \frac{|a-b|}{2}$$
      
      This is a contradiction, so it must be that $a = b$.\\
    
      \item
      By contradiction, we assume $a_n \rightarrow a$, $b_n \rightarrow b$, and $a > b$. Since $a > b$, we set $\epsilon = \frac{a - b}{2}$. There exist a $N \in \mathbb{R}$ where $|a_n - a| < \epsilon \ \ \forall n \geq N$. This can also be written as $a - \epsilon < a_{n} < a + \epsilon$ from Remark 1. Since $b = a - (a - b)$, it implies that $b < a - \frac{a - b}{2}$ or $b < a - \epsilon$. Since $a_n \leq b_n \ \forall n \geq N$ and $a - \epsilon < a_n$, we see that $b < a - \epsilon< a_n \leq b_n$. Since $a - \epsilon < b_n$ and $b_n - b = |b_n - b|$ from $b_n > b$: 
      \begin{align*}
          a - \epsilon - b &< b_n - b  \\
          a - b - \frac{a - b}{2} = \frac{a - b}{2} &< b_n - b \\
          \epsilon &< |b_n - b| 
      \end{align*}
      
      Thus, we see that $|b_n -b| > \epsilon \ \ \forall \ n \geq N$. This means that there is no such $M\in \mathbb{R}$ such that $|b_n - b| < \epsilon \ \ \forall n \geq M$, as there exists within a $n \geq N$ where $|b_n -b| > \epsilon$. This contradicts $b_n \rightarrow b$, so it must be that $a \leq b$.\\
      
      \item 
      By contradiction, we assume that $x \notin [a, b]$. Then, either $x < a$ or $b < x$. \\

      For $x < a$, set $\epsilon = \frac{a-x}{2}$. Because $x_n \rightarrow x$, there exist a $N \in \mathbb{N}$ where for all $n \geq N$, $|x_n - x| < \epsilon$. From Remark 1, it implies that $x - \epsilon < x_n < x + \epsilon$ thus $x_n <  x + \frac{a - x }{2}$. Since $x_n \in [a,b]$, we get that $x_n \geq a$. Since $a > x + \frac{a-x}{2} > x$, we also get that $x_n > x + \frac{a - x }{2}$. The signs of $x_n$ and $x + \frac{a - x }{2}$ are conflicting thus a contradiction. \\

      For $ b< x$, set $\epsilon = \frac{x-b}{2}$. Because $x_n \rightarrow x$, there exist a $N \in \mathbb{N}$ where for all $n \geq N$, $|x_n - x| < \epsilon$. From Remark 1, it implies that $x - \epsilon < x_n < x + \epsilon$ thus $ x - \frac{x-b}{2} < x_n$. Since $x_n \in [a,b]$, we get that $x_n \leq b$. Since $b < x - \frac{x-b}{2} < x$, we also get that $x_n < x - \frac{x - b }{2}$. The signs of $x_n$ and $x - \frac{x - b}{2}$ are conflicting thus a contradiction. \\

      Since both cases result in a contradiction, it must be that $x \in [a, b]$.\\

      \item 
      Since $a_n \rightarrow a$ and $c_n \rightarrow a$, let $\epsilon > 0$, there exist a $N_1, N_2 \in \mathbb{R}$ where $|a_{n_1} - a| < \epsilon \ \ \forall n_1 \geq N_1$ and  $|c_{n_2} - a| < \epsilon \ \ \forall n_2 \geq N_2$. We then denote $N = max\{N_1, N_2\}$. Since $N_1 \leq N$ and $N_2 \leq N$, for all $n \geq P$, it follows that $|a_n - a| < \epsilon$ and  $|c_n - a| < \epsilon$. By Remark 1 and that $a_n \leq c_n$, we see that $a - \epsilon < a_n \leq c_n < a + \epsilon$. Since $a_n \leq b_n \leq c_n$, we get $a - \epsilon < a_n \leq b_n \leq c_n < a + \epsilon$, so $a - \epsilon < b_n < a + \epsilon$, and $ - \epsilon < b_n - a< \epsilon$. This gives us $|b_n - a| < \epsilon \ \ \forall n \geq N$. Since $\epsilon > 0$ was arbitrary, we proved $b_n \rightarrow a$.

      \newpage
    \end{enumerate}

\begin{remark}
Let $a_n$ be a series where for all $n \in \mathbb{N}$, $a_n$ = c for some real number c. It implies that $|a_n - c| = 0 \ \ \forall n \in \mathbb{N}$. Thus, let $\epsilon > 0$ be arbitrary, $|a_n - c| < \epsilon$ for all $n \geq \mathbb{N}$ for any arbitrary $N \in \mathbb{R}$. This proves $a_n \rightarrow c$. 
\end{remark}
    
  \item 
    \begin{enumerate}[label=\alph*)]
      \item 
      We denote the sequence $\hat{a}_n$ where $\forall n \in \mathbb{N}$, $\hat{a}_n = a$, so $\hat{a}_n \rightarrow a$ from Remark 2. Meanwhile, for all $n \in \mathbb{N}$, $b_n = a_j$ for some $j \leq n$ as $a_j$ exists among $a_1, \dots, a_n$. Since $a_j \leq a$, it is implied that $b_n \leq a$. Because of the maximality of $a_j$ in $\{a_1, ..., a_n\}$, $a_j \geq a_n$, so $b_n \geq a_n$. Thus, we get for all $n \in \mathbb{N}$, $a_n \leq b_n \leq \hat{a}_n$. Since $a_n \rightarrow a$ and $\hat{a}_n \rightarrow a$, by Squeeze Theorem, $b_n \rightarrow a$ as desired. \\
      
      \item 
      Yes, the assumption $a_n \leq a$ is necessary otherwise the claim would be false. \\

      For example, we denote $a_n = \frac{1}{n}$ where $a_n \rightarrow 0$, so $a_n > a$. If $b_n = max\{\frac{1}{1}, \dots,\frac{1}{n}\}$, then for all $n \in \mathbb{N}$, $b_n = 1$ since 1 is the largest term in each set, thus $b_n \rightarrow 1$ from Remark 2. Since $0 \neq 1$, the claim fails in this case. Thus, the assumption is necessary.
    \end{enumerate}
\end{enumerate}

\end{document}