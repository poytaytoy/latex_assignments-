\documentclass{article}
\usepackage{enumitem}
\usepackage{mathtools,amssymb,amsthm} % imports amsmath
\usepackage{enumitem}
\usepackage[a4paper, total={6in, 9in}]{geometry}
\newtheorem{remark}{Remark}

\begin{document}

\begin{enumerate}

\item 

\begin{enumerate}
    \item 
    We begin by noting that: 
    $$0 + 0 = 0 \in S + I$$
    $$1 + 0 = 1 \in S + I$$

    Let $s_x + a_x, s_y + a_y \in S + I$. We note that $-s_x \in S$ and $-1 \cdot a_x \in I$. Thus: 

    $$-s_x + (-a_x) = -(s_x + a_x) \in S + I$$

    For addition, we get that: 

    $$s_x + a_x + s_y + a_y = (s_x + s_y) + (a_x + a_y) \in S + I$$

    For multiplication, we get that: 

    $$(s_x + a_x) \cdot (s_y + a_y) = (s_x s_y) + (s_ya_x + s_x a_y + a_x a_y) \in S + I$$

    Thus, we conclude that $S + I$ is a subring. \\

    \item
    We begin by noting that $0 \in I$ and $0 \in S$, so $0 \in S \cap I$. Let $a, b \in S \cap I$. We note that $a + b \in S$ and $a + b \in I$. Thus, $a + b \in S \cap I$. Lastly, let $s \in S$. We note that $as \in S$ and $as \in I$, so $as \in S \cap I$. This proves $S \cap I$ is an ideal of $S$. \\


    \item 
    We first note that $S \subseteq S + I$. Thus, the natural projection map $\pi: S \rightarrow (S+ I)/I$ where $s \mapsto [s]$ is a ring homomorphism. \\
    
    We then note that for all $x \in (S + I)/I$, there exists a $s + a$ such that $s \in S$ and $a \in I$ where $x = [s + a] = [s] + [a] = [s]$ because $I \mid a$, which implies $[a] = [0]$. Thus, $x = \pi (s)$. Hence, $\pi$ is surjective, so it implies $\text{im}(\pi) = (S + I)/(I)$. \\
    
    For $x \in S \cap I$, $\pi(x) = [x] = [0]$ as $I \mid x$, so $x \in \text{ker}(\pi)$. Meanwhile, for $x \in \text{ker}(\pi)$, we note that $[x] = [0]$, which implies $I \mid x$ or $x \in I$. Thus, we get $x \in S \cap I$ and that $\text{ker}(\pi) = S \cap I$. \\
    
    By the First Isomorphism Theorem, we get that $S / (S \cap I) \cong (S + I)/I$. 

\end{enumerate}

\newpage

\item
\begin{enumerate}
    \item 
    We note that $[0] \in J'$, so $0 \in J$. For $a, b \in J$, we get that $[a], [b] \in J'$, so $[a + b] \in J'$ and $a + b \in J$. For $r \in R$, note that $[r] \in R/I$, so $[ra] \in J'$ thus $ra \in J$. This proves $J$ is an ideal of $R$. \\

    \item 
    We note that $0 \in J$, so $[0] \in J/I$. For $[a], [b] \in J/I$, we note that $a + b \in J$, so $[a + b] = [a] + [b] \in J/I$. For $[r] \in R/I$, we note that $ra \in J$, so $[ra] = [r] \cdot [a] \in J/I$. This proves $J/I$ is an ideal of $R/I$. \\

    \item
    We note that the natural projection map of $\pi: R \mapsto R/I$ where $r \mapsto [r]_I$ is a ring homomorphism. Meanwhile, the natural projection map of $\hat{\pi}: R/I \rightarrow (R/I)/(J/I)$ where $[r]_I \mapsto [[r]_I]_{J/I}$ is a ring homomorphism. Hence, if we denote $\varphi = \hat{\pi} \circ \pi$, we get that $\varphi: R \mapsto (R/I)/(J/I)$ where $r \mapsto [[r]]$ is a ring homomorphism. \\

    For all $x \in (R/I)/(J/I)$, there exists an $[r] \in R/I$ where $x = [[r]]$ and consequently, an $r \in R$ where $[[r]] = \varphi(r)$. This proves $varphi$ is surjective, so $\text{im}{(\varphi)} = (R/I)/(J/I)$. \\

    For $x \in J$, $\varphi(x) = [[x]]$. We also note that $[x] \in J/I$, so $[[x]] = [[0]]$ thus $x \in \text{ker}(\varphi)$. Meanwhile, for $x \in \text{ker}(\varphi)$, $\varphi(x) = [[0]]$, so $[x] \in J/I$, which further implies $x \in J$. Thus, $\text{ker}(\varphi) = J$. \\

    By the First Isomorphism Theorem, $R/J \cong (R/I) / (J/I)$. 

\end{enumerate}

\newpage 

\item
\begin{enumerate}
    \item 
    A proper ideal $I$ is prime ideal if and only if $ab \in I$ then $a \in I$ or $b \in I$\\

    \item
    Assume $(r)$ is a prime ideal. By contradiction, we assume $r$ is not prime. Then there exists $a, b \in R$ where $r \mid ab$ but $r \nmid a$ and $r \nmid b$. We note that $ab \in (r)$. This implies either $a \in (r)$ or $b \in (r)$. If we assume $a \in (r)$, then there exists a $q \in R$ where $rq = a$, but that would mean $r \mid a$, a contradiction. Thus, if $(r)$ is a prime ideal, then $r$ is prime. \\

    Assume r is prime. By contradiction, we assume $(r)$ is not a prime ideal, so there exists an $ab \in (r)$ where $a,b \notin (r)$. Since $ab \in (r)$, there exists a $q \in R$ where $ab = rq$, so $r \mid ab$. This implies either $r \mid a$ or $r \mid b$. We assume $r \mid a$, so there exists a $q' \in R$ where $q'r = a$. However, this implies $a \in (r)$, a contradiction. Thus, if $r$ is prime, then $(r)$ is a prime ideal. \\

    \item
    Assume $I$ is a prime ideal of $R$. By contradiction, we assume $R/I$ is not an integral domain, so there exists $[a], [b] \neq [0]$ and $[ab] = [0]$. Since $[ab] = [0]$, $I \mid ab$. This implies either $I\mid a$ or $I \mid b$, so either $[a] = [0]$ or $[b] = [0]$, a contradiction. Thus, if $I$ is a prime ideal, $R/I$ is an integral domain. \\

    Assume $R/I$ is an integral domain. By contradiction, $I$ is not a prime ideal of $R$. Thus, there exists $a,b \notin I$ but $I \mid ab$. This implies $[ab] = [0]$. However, $R/I$ is an integral domain, so either $[a] = [0]$ or $[b] = [0]$. But that implies either $I \mid a$ or $I \mid b$, a contradiction. Thus, if $R/I$ is an integral domain, then $I$ is a prime ideal of $R$.\\

    \item
    We note that $\mathbb{Z}$ is a PID. Thus, for all prime ideals $I$ of $\mathbb{Z}$, there exists an $x \in \mathbb{Z}$ where $I$ = (x). From b), x must be prime. By Euclid's Lemma, all prime numbers are prime, so their principal ideals are also prime ideals. However, 0 satisfies the definition of prime because $\mathbb{Z}$ is an integral domain, so if $0 \mid ab$ then either $a$ or $b$ must be zero, so (0) is also a prime ideal. Thus, all prime ideals of $\mathbb{Z}$ are principal ideals of prime numbers and 0. \\

    \item
    From 1b), we note that $S \cap I$ is an ideal. For $a, b \in S$, if $ab \in S \cap I$, then $ab \in I$, which implies either $a \in I$ or $b \in I$. In other words, we get that either $a \in S \cap I$ or $b \in S \cap I$. Thus, $S \cap I$ is a prime ideal.
\end{enumerate}

\newpage 

\item
\begin{enumerate}
    \item 
    If $[x]^2 = [x]$ for $0 \leq x \leq 2024$, then it implies $[x^2 - x] = [0]$ or $2025 \mid x(x-1)$. We then note that $2025 = 81 \cdot 25$ and that $\gcd(81, 25) = 1$. Thus, we can apply Theorem 7.11, where $m = 81$ and $n = 25$ and note that: 
    \begin{align*}
        [x(x-1)]_{2025} &\mapsto [x(x-1)]_{81} \times [x(x-1)]_{25}\\
        [0] &\mapsto [0]_{81} \times [0]_{25}
    \end{align*}

    We also note that this map is a ring homomorphism, so since $[x^2 - x] = [0]$, it implies that $[x(x-1)] = [0]_{81}$ and $[x(x-1)] = [0]_{25}$. In other words, we get that $81 \mid x(x-1)$ and $25 \mid x(x-1)$. We then note that 25 is a prime power of $5^2$. By Euclid's Lemma, either $5 \mid x$ or $5 \mid x - 1$. Since $\gcd(x - 1, x) = 1$, only one of the factors could be divisible by 5 and will be the one also divisible by 25. A similar argument can be applied that only one of the factors is divisible by 81. Thus, we get that either and $25 \mid x$ or $25 \mid x - 1$ and $81 \mid x$ or $81 \mid x - 1$. This gives us 4 possible combinations. \\

    \textbf{Case 1} If $25 \mid x$ and $81 \mid x$, since 81 and 25 are co-prime, we get that $2025 \mid x$. The only x that satisfies this is if $x = 0$. \\

    \textbf{Case 2} If $25 \mid x-1$ and $81 \mid x-1$, since 81 and 25 are co-prime, we get that $2025 \mid x-1$. The only x that satisfies this is if $x - 1 = 0$ or $x = 1$. \\

    \textbf{Case 3} If $25 \mid x-1$ and $81 \mid x$, it implies there exist $a, b \in \mathbb{Z}$ where $25a = x-1$ and $81b = x$. Thus: 
    \begin{align*}
        25a &= 81b - 1 \\
        1 &= 81b + 25(-a)
    \end{align*}

    We apply the Division Algorithm strategy back in Claim 2.7 to compute that $b = 21$, so $x = 21 \cdot 81 = 1701$. \\ 

    \textbf{Case 4} If $25 \mid x$ and $81 \mid x-1$, it implies there exist $a, b \in \mathbb{Z}$ where $25a = x$ and $81b = x - 1$. Thus: 
    \begin{align*}
        81b &= 25a - 1 \\
        1 &= 25a + 81(-b)
    \end{align*}

    We apply the same strategy to compute that $a = 13$ thus $x = 13 \cdot 25 = 325$ \\ 

    Hence, there are 4 idempotent elements in $\mathbb{Z} / 2025 \mathbb{Z}$. \\

    \item 
    We note that $0^2 = 0$ and $1^2 = 1$, so $0, 1 \in S$. For $a,b \in S$, we note that: 
    \begin{align*}
        (a + b)^2 &= a^2 + 2ab + b^2 \\
        &= a^2 = 2 \cdot 1 \cdot ab + b^2 \\
        &= a^2 + b^2 \\
        &= a + b 
    \end{align*}

    Thus, $a + b \in S$. Meanwhile:: 
    \begin{align*}
        (ab)^2 &= a^2 b^2 \\
        &= ab
    \end{align*}

    Thus, $ab \in S$. Lastly, we note that: 
    \begin{align*}
        a + a &= 2a \\
        &= 2 \cdot 1 \cdot a \\
        &= 0
    \end{align*}

    Thus, we note that $-a = a$ and since $a \in S$, we get that $-a \in S$. We proved $S$ is a subring of $R$. \\

    \item 
    We first note that $(0) = \{r0 : r \in R\} = \{0\}$. Meanwhile, we note that the map $\varphi: R \rightarrow R$ where $r \mapsto r$ is a ring homomorphism. Meanwhile, the $\text{im}(\varphi) = R$ and that $\text{ker}(\varphi) = \{0\} = (0)$. By the First Isomorphism Theorem, we get that $R/(0) \cong R$. \\

   For $(e) + (1-e)$, we note that for all $r \in R$ that 
   
   $$er + (1-e)r = 1r = r$$
   
   Hence, $r \in (e) + (1 -e)$ and $(e) + (1 - e) = R$. This allows us to apply Theorem 8.24 to get that $R/((e)(1-e)) \cong R/(e) \times R/(1-e)$. We then note that for any $a, b \in R$, we get that 
   
   $$(1-e)a\cdot(e)b = (e-e^2)ab = 0ab = 0$$

   This implies that any finite sum in the form of $\sum (e)a_i(1-e)b_i$ is a sum of finitely many zeros, which sums to zero. Hence, $(1-e)(e) = \{0\} = (0)$ and we get that $R \cong R/(0) \cong R/(e) \times R/(1-e)$ or $R \cong R/(e) \times R/(1-e)$ as desired. \\

    \item 
    e
    
\end{enumerate}

\end{enumerate}

\end{document}