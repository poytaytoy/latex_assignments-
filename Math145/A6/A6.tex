\documentclass{article}
\usepackage{enumitem}
\usepackage{mathtools,amssymb,amsthm} % imports amsmath
\usepackage{enumitem}
\usepackage[a4paper, total={6in, 9in}]{geometry}
\newtheorem{remark}{Remark}

\begin{document}

\begin{enumerate}

\item 

\begin{enumerate}
    \item 
    We begin by noting that: 
    $$0 + 0 = 0 \in S + I$$
    $$1 + 0 = 1 \in S + I$$

    Let $s_x + a_x, s_y + a_y \in S + I$. We note that $-s_x \in S$ and $-1 \cdot a_x \in I$. Thus: 

    $$-s_x + (-a_x) = -(s_x + a_x) \in S + I$$

    For addition, we get that: 

    $$s_x + a_x + s_y + a_y = (s_x + s_y) + (a_x + a_y) \in S + I$$

    For multiplication, we get that: 

    $$(s_x + a_x) \cdot (s_y + a_y) = (s_x s_y) + (s_ya_x + s_x a_y + a_x a_y) \in S + I$$

    Thus, we conclude that $S + I$ is a subring. \\

    \item
    We begin by noting that $0 \in I$ and $0 \in S$, so $0 \in S \cap I$. Let $a, b \in S \cap I$. We note that $a + b \in S$ and $a + b \in I$. Thus, $a + b \in S \cap I$. Lastly, let $s \in S$. We note that $as \in S$ and $as \in I$, so $as \in S \cap I$. This proves $S \cap I$ is an ideal of $S$. \\


    \item 
    We first note that $S \subseteq S + I$. Thus, the natural projection map $\pi: S \rightarrow (S+ I)/I$ where $s \mapsto [s]$ is a ring homomorphism. \\
    
    We then note that for all $x \in (S + I)/I$, there exists a $s + a$ such that $s \in S$ and $a \in I$ where $x = [s + a] = [s] + [a] = [s]$ because $I \mid a$, which implies $[a] = [0]$. Thus, $x = \pi (s)$. Hence, $\pi$ is surjective, so it implies $\text{im}(\pi) = (S + I)/(I)$. \\
    
    For $x \in S \cap I$, $\pi(x) = [x] = [0]$ as $I \mid x$, so $x \in \text{ker}(\pi)$. Meanwhile, for $x \in \text{ker}(\pi)$, we note that $[x] = [0]$, which implies $I \mid x$ or $x \in I$. Thus, we get $x \in S \cap I$ and that $\text{ker}(\pi) = S \cap I$. \\
    
    By the First Isomorphism Theorem, we get that $S / (S \cap I) \cong (S + I)/I$. 

\end{enumerate}

\newpage

\item
\begin{enumerate}
    \item 
    We note that $[0] \in J'$, so $0 \in J$. For $a, b \in J$, we get that $[a], [b] \in J'$, so $[a + b] \in J'$ and $a + b \in J$. For $r \in R$, note that $[r] \in R/I$, so $[ra] \in J'$ thus $ra \in J$. This proves $J$ is an ideal of $R$. \\

    \item 
    We note that $0 \in J$, so $[0] \in J/I$. For $[a], [b] \in J/I$, we note that $a + b \in J$, so $[a + b] = [a] + [b] \in J/I$. For $[r] \in R/I$, we note that $ra \in J$, so $[ra] = [r] \cdot [a] \in J/I$. This proves $J/I$ is an ideal of $R/I$. \\

    \item
    We note that the natural projection map of $\pi: R \mapsto R/I$ where $r \mapsto [r]_I$ is a ring homomorphism. Meanwhile, the natural projection map of $\hat{\pi}: R/I \rightarrow (R/I)/(J/I)$ where $[r]_I \mapsto [[r]_I]_{J/I}$ is a ring homomorphism. Hence, if we denote $\varphi = \hat{\pi} \circ \pi$, we get that $\varphi: R \mapsto (R/I)/(J/I)$ where $r \mapsto [[r]]$ is a ring homomorphism. \\

    For all $x \in (R/I)/(J/I)$, there exists an $[r] \in R/I$ where $x = [[r]]$ and consequently, an $r \in R$ where $[[r]] = \varphi(r)$. This proves $varphi$ is surjective, so $\text{im}{(\varphi)} = (R/I)/(J/I)$. \\

    For $x \in J$, $\varphi(x) = [[x]]$. We also note that $[x] \in J/I$, so $[[x]] = [[0]]$ thus $x \in \text{ker}(\varphi)$. Meanwhile, for $x \in \text{ker}(\varphi)$, $\varphi(x) = [[0]]$, so $[x] \in J/I$, which further implies $x \in J$. Thus, $\text{ker}(\varphi) = J$. \\

    By the First Isomorphism Theorem, $R/J \cong (R/I) / (J/I)$. 

\end{enumerate}

\newpage 

\item
\begin{enumerate}
    \item 
    A proper ideal $I$ is prime ideal if and only if $ab \in I$ then $a \in I$ or $b \in I$\\

    \item
    Assume $(r)$ is a prime ideal. By contradiction, we assume $r$ is not prime. Then there exists $a, b \in R$ where $r \mid ab$ but $r \nmid a$ and $r \nmid b$. We note that $ab \in (r)$. This implies either $a \in (r)$ or $b \in (r)$. If we assume $a \in (r)$, then there exists a $q \in R$ where $rq = a$, but that would mean $r \mid a$, a contradiction. Thus, if $(r)$ is a prime ideal, then $r$ is prime. \\

    Assume r is prime. By contradiction, we assume $(r)$ is not a prime ideal, so there exists an $ab \in (r)$ where $a,b \notin (r)$. Since $ab \in (r)$, there exists a $q \in R$ where $ab = rq$, so $r \mid ab$. This implies either $r \mid a$ or $r \mid b$. We assume $r \mid a$, so there exists a $q' \in R$ where $q'r = a$. However, this implies $a \in (r)$, a contradiction. Thus, if $r$ is prime, then $(r)$ is a prime ideal. \\

    \item
    Assume $I$ is a prime ideal of $R$. By contradiction, we assume $R/I$ is not an integral domain, so there exists $[a], [b] \neq [0]$ and $[ab] = [0]$. Since $[ab] = [0]$, $I \mid ab$. This implies either $I\mid a$ or $I \mid b$, so either $[a] = [0]$ or $[b] = [0]$, a contradiction. Thus, if $I$ is a prime ideal, $R/I$ is an integral domain. \\

    Assume $R/I$ is an integral domain. By contradiction, $I$ is not a prime ideal of $R$. Thus, there exists $a,b \notin I$ but $I \mid ab$. This implies $[ab] = [0]$. However, $R/I$ is an integral domain, so either $[a] = [0]$ or $[b] = [0]$. But that implies either $I \mid a$ or $I \mid b$, a contradiction. Thus, if $R/I$ is an integral domain, then $I$ is a prime ideal of $R$.\\

    \item
    We note that $\mathbb{Z}$ is a PID. Thus, for all prime ideals $I$ of $\mathbb{Z}$, there exists an $x \in \mathbb{Z}$ where $I$ = (x). From b), x must be prime. By Euclid's Lemma, all prime numbers are prime, so their principal ideals are also prime ideals. However, 0 satisfies the definition of prime because $\mathbb{Z}$ is an integral domain, so if $0 \mid ab$ then either $a$ or $b$ must be zero, so (0) is also a prime ideal. Thus, all prime ideals of $\mathbb{Z}$ are principal ideals of prime numbers and 0. \\

    \item
    From 1b), we note that $S \cap I$ is an ideal. For $a, b \in S$, if $ab \in S \cap I$, then $ab \in I$, which implies either $a \in I$ or $b \in I$. In other words, we get that either $a \in S \cap I$ or $b \in S \cap I$. Thus, $S \cap I$ is a prime ideal.
\end{enumerate}

\newpage 

\item
\begin{enumerate}
    \item 
    If $[x]^2 = [x]$ for $0 \leq x \leq 2024$, then it implies $[x^2 - x] = [0]$ or $2025 \mid x(x-1)$. We then note that $2025 = 81 \cdot 25$ and that $\gcd(81, 25) = 1$. Thus, we can apply Theorem 7.11, where $m = 81$ and $n = 25$ and note that: 
    \begin{align*}
        [x(x-1)]_{2025} &\mapsto [x(x-1)]_{81} \times [x(x-1)]_{25}\\
        [0] &\mapsto [0]_{81} \times [0]_{25}
    \end{align*}

    We also note that this map is a ring homomorphism, so since $[x^2 - x] = [0]$, it implies that $[x(x-1)] = [0]_{81}$ and $[x(x-1)] = [0]_{25}$. In other words, we get that $81 \mid x(x-1)$ and $25 \mid x(x-1)$. We then note that 25 is a prime power of $5^2$. By Euclid's Lemma, either $5 \mid x$ or $5 \mid x - 1$. Since $\gcd(x - 1, x) = 1$, only one of the factors could be divisible by 5 and will be the one also divisible by 25. A similar argument can be applied that only one of the factors is divisible by 81. Thus, we get that either and $25 \mid x$ or $25 \mid x - 1$ and $81 \mid x$ or $81 \mid x - 1$. This gives us 4 possible combinations. \\

    \textbf{Case 1} If $25 \mid x$ and $81 \mid x$, since 81 and 25 are co-prime, we get that $2025 \mid x$. The only x that satisfies this is if $x = 0$. \\

    \textbf{Case 2} If $25 \mid x-1$ and $81 \mid x-1$, since 81 and 25 are co-prime, we get that $2025 \mid x-1$. The only x that satisfies this is if $x - 1 = 0$ or $x = 1$. \\

    \textbf{Case 3} If $25 \mid x-1$ and $81 \mid x$, it implies there exist $a, b \in \mathbb{Z}$ where $25a = x-1$ and $81b = x$. Thus: 
    \begin{align*}
        25a &= 81b - 1 \\
        1 &= 81b + 25(-a)
    \end{align*}

    We apply the Division Algorithm strategy back in Claim 2.7 to compute that $b = 21$, so $x = 21 \cdot 81 = 1701$. \\ 

    \textbf{Case 4} If $25 \mid x$ and $81 \mid x-1$, it implies there exist $a, b \in \mathbb{Z}$ where $25a = x$ and $81b = x - 1$. Thus: 
    \begin{align*}
        81b &= 25a - 1 \\
        1 &= 25a + 81(-b)
    \end{align*}

    We apply the same strategy to compute that $a = 13$ thus $x = 13 \cdot 25 = 325$ \\ 

    Hence, there are 4 idempotent elements in $\mathbb{Z} / 2025 \mathbb{Z}$. \\

    \item 
    We note that $0^2 = 0$ and $1^2 = 1$, so $0, 1 \in S$. For $a,b \in S$, we note that: 
    \begin{align*}
        (a + b)^2 &= a^2 + 2ab + b^2 \\
        &= a^2 = 2 \cdot 1 \cdot ab + b^2 \\
        &= a^2 + b^2 \\
        &= a + b 
    \end{align*}

    Thus, $a + b \in S$. Meanwhile:: 
    \begin{align*}
        (ab)^2 &= a^2 b^2 \\
        &= ab
    \end{align*}

    Thus, $ab \in S$. Lastly, we note that: 
    \begin{align*}
        a + a &= 2a \\
        &= 2 \cdot 1 \cdot a \\
        &= 0
    \end{align*}

    Thus, we note that $-a = a$ and since $a \in S$, we get that $-a \in S$. We proved $S$ is a subring of $R$. \\

    \item 
    We first note that $(0) = \{r0 : r \in R\} = \{0\}$. Meanwhile, we note that the map $\varphi: R \rightarrow R$ where $r \mapsto r$ is a ring homomorphism. Meanwhile, the $\text{im}(\varphi) = R$ and that $\text{ker}(\varphi) = \{0\} = (0)$. By the First Isomorphism Theorem, we get that $R/(0) \cong R$. \\

   For $(e) + (1-e)$, we note that for all $r \in R$ that 
   
   $$er + (1-e)r = 1r = r$$
   
   Hence, $r \in (e) + (1 -e)$ and $(e) + (1 - e) = R$. This allows us to apply Theorem 8.24 to get that $R/((e)(1-e)) \cong R/(e) \times R/(1-e)$. We then note that for any $a, b \in R$, we get that 
   
   $$(1-e)a\cdot(e)b = (e-e^2)ab = 0ab = 0$$

   This implies that any finite sum in the form of $\sum (e)a_i(1-e)b_i$ is a sum of finitely many zeros, which sums to zero. Hence, $(1-e)(e) = \{0\} = (0)$ and we get that $R \cong R/(0) \cong R/(e) \times R/(1-e)$ or $R \cong R/(e) \times R/(1-e)$ as desired. \\

    \item 
    Let $|R| = 2$ where $R = \{0, 1\}$. By Theorem 7.16, $|R| \cdot 1 = 0$. Thus, $\text{char}(R) = 2$ (we note that $\text{char}(R) = 1$ is impossible because it implies $0 = 1$). By Exercise 7.5 (I proved it in HW5 1c), since $\text{char}(R) = |R|$, we get that $R \cong \mathbb{F}_2$. By induction, we assume all finite commutative ring $R$ where every element is idempotent with $2 \leq |R| \leq k$ for $k \in \mathbb{N}$ is isomorphic to a product of $\mathbb{F}_2$. We now assume such ring $R$ where $|R| = k + 1$. \\

    If there exists an $e \in R$ where it is non-zero and non-unit, by c), we get that \\
    $R \cong R/(e) \times R/(1-e)$. For all $r \in R$, we note that:
    \begin{align*}
        [r]_e^2 &= [r^2]_e = [r]_e \\
        [r]_{1-e}^2 &= [r^2]_{1-e} = [r]_{1-e}
    \end{align*}
    Since the natural projections $R \rightarrow R/(e)$ and $R \rightarrow R/(1-e)$ are surjective, we note that all elements in both rings are idempotent. Since both $(e)$ and $(1-e)$ are the kernels of their respective natural projections, by the pigeonhole principle, a non-injective but surjective map implies $|R/(e)|, |R/(1-e)| < |R|$. Hence, by the induction hypothesis, both are isomorphic to a product of $\mathbb{F}_2$. Hence, we get that: 

    $$R \cong (\mathbb{F}_2 \times \cdots \times \mathbb{F}_2 ) \times (\mathbb{F}_2 \times \cdots \times \mathbb{F}_2 ) $$
    $$R \cong \mathbb{F}_2 \times \cdots \times \mathbb{F}_2  $$

    Meanwhile, if there is no non-zero and non-unit element in $R$, then $R$ must be a field because every non-zero element is a unit. For all $a \in R^{\times}$, there exists a $b \in R^{\times}$ where $ab = 1$. This implies $a(ab) = a$, but $a^2b = ab$, so $1 = ab = a$. Thus, $R^{\times} = \{1\}$. Since $R$ is a field, we get that $R = \{0, 1\}$, which contradicts our assumption of $|R| = k + 1$, making it impossible. \\

    We proved that all finite commutative rings $R$ where every element is idempotent is isomorphic to a product of $\mathbb{F}_2$.

    
\end{enumerate}

\newpage

\item
\begin{enumerate}
    \item 
    We first prove that $\text{im}(ev_a)$ is a subring of $\mathbb{C}$. We note that $1, 0 \in \mathbb{Z}[x]$, so $0, 1 \in \text{im}(ev_a)$. For all $d,e \in \text{im}(ev_a)$, there exists a $f, g \in \mathbb{Z}[x]$ where $f(a) = d$ and $g(a) = e$. We note that $-f \in Z[x]$, so $-f(a) = -d \in \text{im}(ev_a)$. Meanwhile, $f + g, fg \in \mathbb{Z}[x]$, so $f(a) + g(a) = d + e, f(a)g(a) = de \in \text{im}(ev_a)$. This proves $\text{im}(ev_a)$ is a subring of $\mathbb{C}$. We also note $x \in \mathbb{Z}[x]$, so $a \in \text{im}(ev_a)$.\\

    We now prove it is the smallest subring containing $a$. For any subring $S$ containing $a$, for all $x \in \text{im}(ev_a)$, there also exists an $f \in \mathbb{Z}[x]$ where $f(a) = x$. $f$ is a polynomial and $S$ is closed under addition and multiplication for all of its elements. We also note $\mathbb{Z} \subseteq S$ because we can add $1, -1 \in S$ and we can add them indefinitely. This implies $f(a) = x \in S$, so $\text{im}(ev_a) \subseteq S$. Since all subrings $S$ containing $a$ contains $\text{im}(ev_a)$, it is the smallest subring containing $a$, which implies $\text{im}(ev_a) = \mathbb{Z}[a]$. \\

    \item 
    For each $\beta_k$, there exists a $f_k \in \mathbb{Z}[x]$ where $f_k(a) = \beta_k$ from our result in (a). We then denote $d = \max\{ \deg(f_1), \cdots , \deg(f_k) \} + 1$. Since $-a^d \in \mathbb{Z}[a]$, there exists, $c_1, \cdots, c_n$ where $c_1f_1(a) + \cdots + c_n f_n(a) = -a^d $. We construct the polynomial: 

    $$f(x) = x^d + c_1f_1(x) + \cdots + c_n f_n(x)$$ 

    We note that $f(a) = 0$. Since $\deg(x^d) \geq \deg(f_k)$ for all $1 \leq k \leq n$, the leading coefficient of $f$ is 1. Thus, we constructed a monic polynomial where $f(a) = 0$. \\

    \item

    We denote $C=\{c_0+c_1a+\cdots+c_{d-1}a^{d-1}:c_0,c_1,\dots,c_{d-1}\in\mathbb{Z}\}$.\\

    For all $c\in\mathbb{Z}[a]$, there exists $f\in\mathbb{Z}[x]$ with $f(a)=c$ as proven in (a). By Proposition 9.4, $f(x)=q(x)g(x)+r(x)$ for $q,r\in\mathbb{Z}[x]$ since $g(x)$ is monic, so its leading coefficient is a unit and $\deg r<\deg g$. Since $g(a)=0$, we get $f(a)=r(a)$. Since $\deg r\le d-1$, we get that $r(a)$ is a sum of integer coefficients up to $a^{d-1}$, so $r(a)=c\in C$.\\

    For all $c\in C$, we have $c=c_0+c_1a+\cdots+c_{d-1}a^{d-1}$. The polynomial $f(x)=c_0+c_1x+\cdots+c_{d-1}x^{d-1}\in\mathbb{Z}[x]$, so $f(a)=c\in\mathbb{Z}[a]$. Hence, $\mathbb{Z}[a]=C$.


\end{enumerate}

\newpage 

\item 
\begin{enumerate}
    \item 
    We note that $(0)^{\infty}_{n=1}, (1)^{\infty}_{n=1}  \in \underset{\leftarrow}{\lim}\, R_n$ because $f_n(0) = 0$ and $f_n(1) = 1$ for all $n \in \mathbb{N}$. Let $(a_n)^{\infty}_{n=1}, (b_n)^{\infty}_{n=1} \in \underset{\leftarrow}{\lim}\, R_n$. We get that $(a_n + b_n)^{\infty}_{n=1} \in \underset{\leftarrow}{\lim}\, R_n$ because $f_n(a_{n+1} + b_{n+1}) = f_n(a_{n+1}) + f_n(b_{n+1}) = a_n+b_n$. We also get that $(a_n b_n)^{\infty}_{n=1} \in \underset{\leftarrow}{\lim}\, R_n$ because $f_n(a_{n+1} b_{n+1}) = f_n(a_{n+1})  f_n(b_{n+1}) = a_nb_n$.  Lastly, we get that $-(a_n)^{\infty}_{n=1} = (-a_n)^{\infty}_{n=1}  \in \underset{\leftarrow}{\lim}\, R_n$ because $f_n(-a_{n+1}) = -f_n(a_{n+1}) = -a_n$. Thus, $\underset{\leftarrow}{\lim}\, R_n$ is a subring of $\prod_{n=1}^{\infty}R_n$. \\

    \item 
    By contradiction, $\mathbb{Z}_p$ does not have characteristic 0. This implies there exist a positive integer $m$ where $m \cdot (1)^{\infty}_{n=1} = (0)^{\infty}_{n=1}$. In other words, for all $n \in \mathbb{N}$, we get that \\
    $m \cdot 1 \equiv 0 \pmod {p^n}$. However, there exist large enough a $k \in \mathbb{N}$ where $p^k > m$, so \\
    $m \not \equiv 0 \pmod {p^k}$. This is a contradiction, so the characteristic of $\text{char}(\mathbb{Z}_p)$ must be 0.  \\

    \item 
    We prove that $\varphi$ is injective. We assume $\varphi((a_n)_{n=1}^{\infty}) = \varphi((b_n)_{n=1}^{\infty}) = (r_n)^{\infty}_{n=1}$ for $(r_n)^{\infty}_{n=1} \in \mathbb{Z}_p$ and $(a_n)_{n=1}^{\infty}, (b_n)_{n=1}^{\infty} \in S^{\mathbb{N}}$. \\

    We prove by induction that $a_n = b_n$ for all $n \in \mathbb{N}$. For $n = 1$, we note that $[a_1]_p = [b_1]_p$, so $a_1 - b_1 \equiv 0 \pmod {p}$. However, $a_1, b_1 \in S$, so $-(p-1) \leq a_1 - b_1 \leq p-1$. The only option for $p \mid (a_1 - b_1)$ is that $a_1 - b_1 = 0$ or $a_1 = b_1$. Let $k \in \mathbb{N}$. We assume that the induction hypothesis holds true for all $1 \leq m \leq k$. For $n = k + 1$, we get that: 

    $$[a_1 + \cdots + a_{k+1}p^{k}]_{p^{k+1}} = [b_1 + \cdots + b_{k+1}p^{k}]_{p^{k+1}}$$ 

    or that: 

    $$(a_1 - b_1) + \cdots + (a_{k+1} - b_{k+1})p^{k} \equiv 0 \pmod {p^{k+1}}$$

    Because of the induction hypothesis that all $1 \leq m \leq k$ has $a_m - b_m = 0$. Thus: 

    $$(a_{k+1} - b_{k+1})p^{k} \equiv 0 \pmod {p^{k+1}}$$

    This implies that $p \mid (a_{k+1} - b_{k+1})$. It follows from the same reasoning from $n=1$ that $a_{k+1} - b_{k+1}$ must be equal to $0$, thus $a_{k+1} = b_{k+1}$. This proves that $(a_n)^{\infty}_{n=1} = (b_n)^{\infty}_{n=1}$ and that $\varphi$ is injective. \\

    We now prove that $\text{im}(\varphi) = \mathbb{Z}_p$. Let $(r_n)_{n=1}^{\infty} \in \text{im}(\varphi)$. We get that $r_{n+1} = [a_1 + \cdots + a_{n+1}p^n]_{p^{n+1}}$. We then note that: 
    
    
    $$f_n(r_{n+1}) = [a_1 + \cdots + a_{n}p^{n-1} + a_{n+1}p^n]_{p^{n}} = [a_1 + \cdots + a_{n}p^{n-1}]_{p^{n}} = r_n$$

    Thus, we proved that $(r_n)_{n=1}^{\infty} \in \mathbb{Z}_p$, so $\text{im}(\varphi) \subseteq \mathbb{Z}_p$. Meanwhile for $(r_n)_{n=1}^{\infty} \in \mathbb{Z}_p$, we note that for all $n \in \mathbb{N}$, we get that $r_n = [x_n]_{p^n}$ where $0 \leq x_n \leq p^n - 1$. We then denote $a_n = \lfloor x_n / p^{n-1} \rfloor$ and note that $0 \leq a_n \leq p - 1$, so $a_n \in S$. We note that 

    $$r_n = a_n p^{n-1} + x_n \% p^{n-1}$$ 
    
    Observe that $f_{n-1}(r_{n}) = [x_n \% p^{n-1}]_{p^{n-1}} = r_{n-1}$, so we get that $x_n \% p^{n-1} = x_{n-1}$. We repeat the same procedure for $r_{n-1}$ and get that $x_{n-1} = a_{n-1}p^{n-2} + x_{n-2}$. We can repeat this until we reach $x_1$ where $x_1 = a_1 \cdot p^0 = a_1 = x_2 \% p$. Hence we get that: 
    
    
    $$r_n = [a_n p^{n-1} + a_{n-1}p^{n-2}+ \cdots + a_1]_{p^{n}}$$

    Note that for all $1 \leq k \leq n$, $a_k \in S$ by our construction. Hence, there exists a \\
    $(a_n)^{\infty}_{n=1} \in S^{\mathbb{N}}$ where $\varphi((a_n)^{\infty}_{n=1}) = (r_n)^{\infty}_{n=1}$. Hence, $(r_n)^{\infty}_{n=1} \in \text{im}(\varphi)$, so $\text{im}(\varphi) \supseteq \mathbb{Z}_p$ and we conclude that $\text{im}(\varphi) = \mathbb{Z}_p$.\\

    \item 
    \textbf{Lemma:} For $(r_n)^{\infty}_{n=1} \in \mathbb{Z}_p$, $(r_n)^{\infty}_{n=1}$ is a unit iff $r_1 \neq 0$.\\

    We assume $r_1 = 0$, so there does not exist an element $a$ in $\mathbb{Z}/p\mathbb{Z}$ where $a \cdot r_1 = 1$ because $a \cdot r_1 = 0$. We note that $1 = (1)^{\infty}_{n=1}\in \mathbb{Z}_p$. Since there does not exist a $(\hat{r}_n)^{\infty}_{n=1} \in \mathbb{Z}_p$ where $r_1 \cdot \hat{r}_1 = 1$, the inverse of $(r_n)^{\infty}_{n=1}$ does not exist, so it is not a unit. Hence, we get that if $(r_n)^{\infty}_{n=1}$ is a unit, then $r_1 \neq 0$. \\

    For the converse, we assume $r_1 \neq 0$. We will construct a $(b_n)^{\infty}_{n=1} \in \mathbb{Z}_p$ where \\ $(r_n)^{\infty}_{n=1} \cdot (b_n)^{\infty}_{n=1} = (1)^{\infty}_{n=1}$ to show that the inverse exists. For $b_1$, we note that $r_1$ is non-zero and since $\mathbb{Z}/p\mathbb{Z}$ is a field, we can denote $b_1 = r_1^{-1}$. We then assume for $b_n$ that $b_n \cdot r_n = 1$. For $b_{n+1}$, we first note that $f_n(r_{n+1}) = r_n$ and $f_n(b_{n+1}) = b_n$. We then note that there exist $r, b \in \mathbb{Z}$ where $0 < r, b < p^n$ and $r_n = [r]_{p^n}$ and $b_n = [b]_{p^n}$. This implies non-negative integers $c, d$ that: 

    $$r_{n+1} = [cp^n + r]_{p^{n+1}}$$
    $$b_{n+1} = [dp^n + b]_{p^{n+1}}$$

    Hence, we get that:  
    \begin{align*}
        r_{n+1} \cdot b_{n+1} &= [cdp^{2n} + drp^n + cbp^n + rb]_{p^{n+1}} \\
        &=  [drp^n + cbp^n + rb]_{p^{n+1}} 
    \end{align*}

    Since $rb \equiv 1 \pmod {p^n}$, there exists a non-negative integer $e$ where $rb = ep^n + 1$. Hence we now get that: 
    \begin{align*}
        r_{n+1} \cdot b_{n+1} &= [drp^n + cbp^n + ep^n + 1]_{p^{n+1}} \\
        &= [p^n(dr + cb + e) + 1]_{p^{n+1}}
    \end{align*}

    To achieve the desired $p^n(dr + cb + e) + 1 \equiv 1 \pmod {p^{n+1}}$, we need to make $p \mid dr + cb + e$. In other words, we can express this as $[dr + cb + e]_p = [0]_p$ in $\mathbb{Z}/p\mathbb{Z}$. We first note that $[r]_p$ is non-zero because $r_1$ is non-zero and $r_1 = f_1 \cdots \circ f_{n-1} \circ f_n (r) = [r]_p$, so $[r]_p^{-1}$ exists. Hence, if we denote $d$ as the positive integer where $[d]_p = [r]_p^{-1} \cdot (-[cb + e])$. We get that $([r]_p^{-1} \cdot (-[cb + e]))[r]_p + [cb + e]_p = [0]_p$ as desired. Hence, by denoting $b_{n+1} = [dp^n + b]_{p^{n+1}}$, we get that $b_{n+1} \cdot r_{n+1} = [(dp^n + b)(cp^n + r)]_{p^{n+1}} = [1]_{p^{n+1}}$. We have inductively created $(b_n)^{\infty}_{n=1}$ that serves as the inverse of $(r_n)^{\infty}_{n=1}$, so we proved $(r_n)^{\infty}_{n=1}$ is a unit. \\

    \textbf{Remark} For $(a_n)^{\infty}_{n=1} \in S^{\mathbb{N}}$ where $\varphi((a_n)^{\infty}_{n=1})$ is equal to $(u_n)^{\infty}_{n=1} \in \mathbb{Z}_p$, we note that $[a_1]_p = u_1$. The Lemma implies that $(u_n)^{\infty}_{n=1}$ is a unit iff $a_1$ is non-zero. \\

    We assume $\nu_p(a) = m$. Let $(a_n)^{\infty}_{n=1} \in S^{\mathbb{N}}$ and $\varphi((a_n)^{\infty}_{n=1}) = a$. By the definition of $\nu_p$, we note that $a_1, \cdots, a_{m}$ are all equal to $0$. Hence, we note that:

    $$\sum^{\infty}_{n= 1} a_n p^{n-1} = p^m \cdot \sum^{\infty}_{n= m + 1} a_n p^{n-m-1}$$ 

    Hence, we denote the series $(a_{n+m})^{\infty}_{n=1}$. Since $a_{m+1}$ is non-zero, which is the first term of the series, we note that $\sum^{\infty}_{n= m + 1} a_n p^{n-m-1}$ is a unit from the Remark. Thus, $a = p^m u$ for some unit $u \in \mathbb{Z}_p$. \\ 

    For the converse, we assume $a = p^m u$ where $u$ is a unit in $\mathbb{Z}_p$. We note that we can express $u$ as

    $$u = \sum^{\infty}_{n=1}a_n p^{n-1}$$

    where $a_1$ is non-zero from the Remark since $u$ is a unit. If we multiply $p^m$, we note that it is equivalent to: 

    $$p^m u = p^m \cdot \sum^{\infty}_{n=1}a_n p^{n-1} = \sum^{\infty}_{n=1}a_n p^{n + m-1} $$

    Hence, we denote a $(b_n)^{\infty}_{n=1} \in S^{\mathbb{N}}$ where for $n \leq m$, we get that $b_n = 0$ and for $n > m$, we get that $b_n = a_{n - m}$. Hence, 

    $$\sum^{\infty}_{n = 1}b_n p^{n-1} = \sum^{m}_{n = 1}0 p^{n-1} + \sum^{\infty}_{n=1}a_n p^{n + m-1} = p^m u$$

    Hence, we get that $\nu_p(a) = \nu(p^mu) = \nu_p(\sum^{\infty}_{n = 1}b_np^{n-1}) = m+1 - 1 = m$ as desired as the first non-zero term is $b_{m+1}$. 
    \\ 
    \item 
    We first prove that $\mathbb{Z}_p$ is an integral domain. Let $a, b \in \mathbb{Z}_p$ and both are non-zero. We denote $m = \nu_p(a)$ and $n = \nu_p(b)$. We also get that there exist units $u, v$ where $a = p^m u$ and $b = p^n v$. We then express $u = (u_n)_{n=1}^{\infty}$ and $v = (v_n)_{n=1}^{\infty}$. Since $u_1, v_1 \in \mathbb{Z}/p\mathbb{Z}$, from the Lemma, since $u$ and $v$ are units, $u_1$ and $v_1$ are non-zero, which implies $u_1 \cdot v_1$ is non-zero because $\mathbb{Z}/p\mathbb{Z}$ is a field (thus an integral domain). By the Lemma again, this further implies that $uv$ is a unit. Thus, we have that $ab = p^{m+n}(uv)$, and it follows that $\nu_p(ab) = m + n$. We then denote $(d_n)_{n=1}^{\infty} \in S^{\mathbb{N}}$ where $\varphi((d_n)_{n=1}^{\infty}) = ab$ and we get that $d_{m+n + 1}$ is non-zero. Hence, $(d_n)_{n=1}^{\infty} \neq (0)_{n=1}^{\infty}$, so $ab \neq 0$. This proves that $\mathbb{Z}_p$ is an integral domain.\\

    We can now prove that it is a Euclidean domain. We first denote $\nu_p$ as the $N(x)$ tied to $\mathbb{Z}_p$. Let $a, b \in \mathbb{Z}_p$ where $a \neq 0$. We consider the cases as follows with $q, r \in \mathbb{Z}_p$ for the form $b = aq + r$:\\

    \textbf{Case 1}: If $b = 0$, then it follows that $q = 0$ and $r = 0$ where $0 = a0 + 0 = 0$. \\

    For the remaining cases, we assume $b \neq 0$. Thus, we can denote $\nu_p(b) = m$ and $\nu_p(a) = n$ and that $b = p^m u$ and $a = p^n v$ with $u, v$ being units in $\mathbb{Z}_p$.\\

    \textbf{Case 2}: If $m > n$, then it follows that $r = 0$ and $q = p^{m-n}(v^{-1}u)$. Thus, $b = (p^n v)(p^{m-n}(v^{-1}u)) = p^m u = b$. (Take notice that we had earlier shown with the integral domain proof that the multiplication between two arbitrary units in $\mathbb{Z}_p$ is still a unit, so $v^{-1}u$ is a unit.) \\

    \textbf{Case 3}: If $m = n$, then it follows that $r = 0$ and $q = (v^{-1}u)$. Thus, $b = (p^n v)((v^{-1}u)) = p^m u = b$. \\

    \textbf{Case 4}: If $m < n$, then it follows that $r = p^m u$ and $q = 0$. Thus, $b = (p^n v)0 + p^m u = p^m u = b$. Since $\nu_p(r) = m$, we get that $\nu_p(r) < \nu_p(a)$. \\

    We showed that in all cases, it is either $r = 0$ or $\nu_p(r) < \nu_p(a)$. This proves that $\mathbb{Z}_p$ is a Euclidean domain.


\end{enumerate}

\end{enumerate}

\end{document}