\documentclass{article}
\usepackage{enumitem}
\usepackage{mathtools,amssymb,amsthm} % imports amsmath
\usepackage{enumitem}
\usepackage[a4paper, total={6in, 9in}]{geometry}
\newtheorem{remark}{Remark}

\begin{document}

\begin{enumerate}
\item
\begin{enumerate}
    \item 
    We note that F is a field because $f(x)$ is an irreducible polynomial in $\mathbb{F}_p$. Due to the inclusion of contants in $\mathbb{F}_p[x]$ and that $\deg f \geq 1$, we get that $\mathbb{F}_p \subset F$. Hence, $f(x) = a_0 + a_1x + \cdots + a_dx^d$ with $a_i \in \mathbb{F}_p$. We note that:
    
    $$f(\beta)^p = (a_0 + a_1\beta + \cdots + a_d\beta^d)^p = a_0^p + a_1^p\beta^p + \cdots + a_d^p\beta^{dp}$$

    Because of $1 \in \mathbb{F}_p \subset F$, $char (\mathbb{F}_p) = p = char(F)$, and that $F$ is a field, the result follows from Frobenius Map of $(a + b)^p \mapsto a^p + b^p$. Hence, we note that $f(\beta)^p = f(\beta^p) = 0^p = 0$. This concludes that $\beta^p$ is a root of $f(x)$ in $F$. \hfill $\square$  \\

    
    \item 

    We note that since $\alpha = [x]$, we get that $f(\alpha) = [f(x)] = 0$. Hence $\alpha$ is a root of $f(x)$ in $F$. From (a), it follows that $\{\alpha, \alpha^p, \cdots, \alpha^{p^{d-1}} \}$ are all roots of $f(x)$. \\

    \textbf{Claim: } $\{\alpha, \alpha^p, \cdots, \alpha^{p^{d-1}} \}$ are distinct \\ 

    By contradiction, there exist a $0 \leq i < j \leq d - 1$ where $\alpha^{p^i} = \alpha^{p^j}$. Hence, $\alpha^{p^j} - \alpha^{p^i} = (\alpha^{p^{j-i}} - \alpha)^{p^i} = 0$ by applying the Frobenius Map $a^{p^i} + b^{p^i} \mapsto (a + b)^{p^i}$. Thus, we get that $[x^{p^{j-i}} - x] = 0$ as we recall that $\alpha = [x]$. This implies that $f(x) \mid x^{p^{j-i}} - x$. By Theorem 9.23, $x^{p^{j-i}} - x$ is the product of all monic irreducible polynomials in $\mathbb{F}_p[x]$ of degree dividing $j - i$. However, $j - i < d$, so $d \nmid j-i$, so $f(x)$ is not a monic irreducible factor of $x^{p^{j-i}} - x$, which gives us a contradiction. Hence, they must be distinct. \hfill $\square$ \\

    Since we have found $d$ distinct roots for a monic polynomial $f(x)$, by Corollary 9.7, it follows that $\prod^{d-1}_{i=0} (x - \alpha^{p^i}) \mid f(x)$. Since $f(x)$ is monic and the product has degree $d$, we get that $\prod^{d-1}_{i=0} (x - \alpha^{p^i}) = f(x)$ as desired. \hfill $\square$  \\

    \item 
    By Proposition 9.21, defining a homomorphism of $\varphi:F \rightarrow E$ is the same as giving a root $\beta$ of $f(x)$ in $E$, which exists as defined. By the First Isomorphism Theorem, the natural map of $\phi: F / ker(\varphi) \rightarrow im(\varphi)$ is an isomorphism. Note that $F$ is a field, so any homomorphism is injective, which implies that only $\varphi(0) = 0$. Thus, the $ker(\varphi) = \{0\}$, which further implies that $F / ker(\varphi) = F$. We note $im(\varphi)$ is finite from being isomorphic to a finite field and also a subring of a field $E$, so it is a finite integral domain thus a subfield of $E$. Since $\phi$ is an isomorphism, note that $\{\phi(\alpha), \phi(\alpha^p), \cdots, \phi(\alpha^{p^{d-1}}) \} \subset E$ must also be all distinct. \\
    
    We then note that since $char(E) = p$, $\mathbb{F}_p$ is a subfield of E. Since all of $f(x)$'s coefficients lie in $\mathbb{F}_p$ and that $\phi(1) = 1$, for any $a \in \mathbb{F}_p$, $\phi(a) = a$. Hence, $\phi(f(x)) = \phi(a_0 + a_1x + \cdots + a_dx^d) = a_0 + a_1\phi(x) + \cdots + a_d\phi(x)^d = f(\phi(x))$. For any root $\alpha^{p^i} \in F$, we note that $\phi(f(\alpha^{p^i})) = \phi(0) = 0 = f(\phi(\alpha^{p^i}))$. Hence, $\{\phi(\alpha), \phi(\alpha^p), \cdots, \phi(\alpha^{p^{d-1}}) \}$ are all roots of $f(x) \in E$. Since $f(x)$ has $d$ distinct roots, by Corollary 9.7, it must be in the form $c(x - \phi(\alpha))\cdots(x - \phi(\alpha^{p^{d-1}})) $, so it splits completely. \hfill $\square$  \\


\end{enumerate}
\newpage 


\item 
\begin{enumerate}
    \item 
    We assume $o(\alpha) = m$. Since $|\mathbb{F}_{p^n}^{\times}| = p^n - 1$, we get that $\alpha^{p^n - 1} = 1$, so $m \mid p^n - 1$ as desired. \\ 

    We assume $m \mid p^n - 1$. Then for some $\beta \in \mathbb{F}_{p^n}$ by Theorem 10.16, we get that $o(\beta) = |\mathbb{F}_{p^n}^{\times}| = p^n - 1$. This implies that $\mathbb{F_{p^n}} = \{0, \beta, \cdots, \beta^{p^n - 1}\}$. Since $m \mid p^n - 1$, there exist some integer $q$ where $mq = p^n - 1$. Since $q \leq p^n - 1$, $\beta^q \in \mathbb{F}_{p^n}$. Note that $(\beta^q)^m = \beta^{p^n - 1}$, so $o(\beta^q) \mid m$. Suppose $o(\beta^q) < m$ and let $d = o(\beta^q)$, this implies that $\beta^{qd} = 1$. However, $qd < qm = p^n - 1$, so this contradicts $o(\beta)$'s minimality. Hence, it must be that $o(\beta^q) = m$. \hfill $\square$  \\

    \item 
    \textbf{Claim:} $\Phi_m(x)$ splits completely in $\mathbb{F}_{p^d}$. \\

    Since $d = o_m(p)$, we get that $m \mid p^d - 1$. By the result of the proof from Theorem 10.7, if $d \mid |\mathbb{F}_{p^n}^{\times}| $, then the number of elements in $\mathbb{F}_{p^n}$ with order exactly d is $\phi(d)$. Since $m \mid m$, there exist $\phi(m)$ elements with order $m$. Hence, for each of the $\phi(m)$ elements $\beta$ with $o(\beta) = m$, we get that $\Phi_m(\beta) = 0$ from HW7(b). $\Phi_m(x)$ is degree $\phi(m)$ with $\phi(m)$ distinct roots. By Corollary 9.7, $\Phi_m(x)$ is divisible by the product of $\phi(m)$ linear polynomials, so it splits completely in $\mathbb{F}_{p^n}$. \hfill $\square$ \\

    \textbf{Claim:} Irreducible polynomial factors of $\Phi_m(x)$ in $\mathbb{F}_p$ has degree $d$. \\

    Let $f(x)$ be an irreducible polynomial factor of $\Phi_m(x)$ in $\mathbb{F}_p$ with degree $k$. Since $f(x)$ is a factor of $\Phi_m(x)$, which splits completely in $\mathbb{F}_{p^d}$. $f(x)$ must also split completely and shares roots with $\Phi_m(x)$. Earlier, we showed that all roots of $\Phi_m(x)$ has order $m$. Hence, there is a root $\alpha$ of $f(x)$ where $o(\alpha) = m$. \\
    
    By Theorem 9.23, $f(x)$ is factor of $x^{p^k} - x$ in $\mathbb{F}_p$ since $k \mid k$. Since there is a natural way to $\mathbb{F}_p$ in $\mathbb{F}_{p^d}$, we note that $f(x)$ is a factor of $x^{p^k} - x$ in $\mathbb{F}_{p^d}$ and consequently, $\alpha^{p^k} - \alpha = 0$ and that $\alpha^{p^k - 1} = 1$. Since $o(\alpha) = m$, $p^k \equiv 1 \pmod{m}$ and $d \mid k$. \\
    
    Meanwhile, from Theorem 9.21, there exist a homomorphism $\mathbb{F}_p[x]/(f(x)) \rightarrow \mathbb{F}_{p^d}$ is the same as giving the root $\alpha$ in $\mathbb{F}_{p^d}$. Since $\mathbb{F}_p[x]/(f(x))$ is a field with order $p^k$, from Theorem 10.1 (d), this homomorphism implies that $k \mid d$. Since $k \mid d$ and $d \mid k$, we get that $k = d$ as desired. \hfill $\square$  \\

    Since $\Phi_m(x)$ is a monic polynomial, it can be written as a product of irreducible polynomials as quoted from Example 9.15. Since each irreducible polynomial factor is degree $d$ and $\Phi_m(x)$ is degree $\phi(m)$, it must be that there exists $\phi(m) / d$ irreducible polynomial factors for $\Phi_m(x)$. \hfill $\square$  \\

    \item 
    We assume $p$ is not a square mod q. We first note that by Euler Criterion, we get that $\phi(m) = q - 1$. Meanwhile, $(\frac{p}{q}) = -1 \equiv p^{(q-1)/2} \pmod{q}$. This implies $d \nmid (q-1)/2$. We also note that $1 \equiv p^{q-1} \pmod{q}$, so $d \mid q-1$. Since $dr = q-1$, it follows that $d(r/2) = (q-1)/2$. $(r/2) \notin \mathbb{Z}$ or else $d \mid (q-1)/2$. This implies that $r$ is odd. Since $q-1$ is even from $q$ being an odd prime, it follows that $d$ must be even. \\

    We assume $d$ is even and $r$ is odd. Since $rd = q - 1$, we note that $r/2 \notin \mathbb{Z}$ and that $d \nmid (q-1)/2$. This implies that $p^{(q-1)/2} \not \equiv 1 \pmod{q}$. However, $p^{q-1} \equiv 1 \pmod{q}$ by Fermat's Little Theorem, so $p^{(q-1)/2} \equiv -1 \pmod{q}$. By Corollary 11.3, $p$ is a quadratic non-residue mod $q$, so $p$ is not a square mod $q$. \hfill $\square$ 

\end{enumerate}

\newpage 

\item
\begin{enumerate}
    \item 
    For any $a_i \in F$, we re-organize $f(x)$ to $(x-a_i)\prod^n_{\substack{k = 1 \\ k \neq i}}(x-a_k)$. We note that:
    
    $$f'(x) = (x-a_i)(\prod^n_{\substack{k = 1 \\ k \neq i}}(x-a_k))' + 1 \cdot \prod^n_{\substack{k = 1 \\ k \neq i}}(x-a_k) = \prod^n_{\substack{k = 1 \\ k \neq i}}(x-a_k)$$

    Within the product of $f(a_i)$, there exists $i-1$ instances where $i > k$. We correct this by applying $(-1)^{i-1}f'(a_i)$ to get that: 

    $$(-1)^{i-1}f'(a_i) = \prod^{i-1}_{k = 1}(a_k - a_i) \cdot \prod^n_{k = i + 1}(a_i - a_k)$$

    We then note that for any $1 \leq i < j \leq n$, $(a_i - a_j)$ is a factor in the products of $(-1)^{i-1}f'(a_i)$ and $(-1)^{j-1}f'(a_j)$. Hence, we note that the product of all such $(-1)^{i-1}f'(a_i)$ is equal to:

    \begin{align*}
        \prod^n_{i=1}(-1)^{i-1}f'(a_i) &= (-1)^{0 + \cdots + n - 1}\prod^n_{i=1}f'(a_i) \\
        &= (-1)^{n(n-1)/2}\prod_{1 \leq i < j \leq n}(a_i - a_j)^2\\
        &= \triangle f
    \end{align*}

    as desired. \hfill $\square$ \\

    \item 
    Let $f(x) = x^8 + x^2 + 1$. We note that $f'(x) = 8x^7 + 2x = 2x(4x^6 + 1)$ We then note that if $f(x)$ splits completely in F, it must have 8 roots because $\deg f = 8$. We apply the altered calculation of the determinant from (a) and consider that $(-1)^(8\cdot 7/2) = (-1)^{28} = 1$. Thus: 

    $$\triangle f = \prod^8_{i=1}f'(a_i) = 2^8 \cdot \prod^8_{i=1}a_i\cdot\prod^8_{i=1}4a_i^6 + 1$$


    \textbf{Claim (Vieta):} If a polynomial $f(x)$ splits completely and is monic. Then $f(0)$ is equal to product of its roots times $(-1)^{\deg f}$. \\

    Let us assume $f(x) = (x - a_1) \cdots (x-a_k)$ with degree $k$. Then $f(0) = (-a_1) \cdots (-a_k) = (-1)^k \cdot a_1 \cdots a_k$. \hfill $\square$  \\

    With Vieta, we note that $f(0) = 1$, so by Vieta, we get that $(-1)^8\prod^{8}_{i=1} a_i = \prod^{8}_{i=1} a_i = 1$. Thus, we can simplify $\triangle f$ into: 
    
    $$\triangle f = 2^8 \cdot\prod^8_{i=1}4a_i^6 + 1$$

    We also then note that for any root $a_i$, $a_i^8 + a_i^2 + 1 = 0$, so $a_i^6 = -(1 + 1\cdot a_i^{-2})$ Thus: 

    \begin{align*}
        \triangle f &= 2^8 \cdot\prod^8_{i=1}-4(1 + 1\cdot a_i^{-2}) + 1\\
        &= 2^8 \cdot\prod^8_{i=1}-4 - 4\cdot a_i^{-2} + 1\\
        &= 2^8 \cdot \prod^8_{i=1}-a_i^{-2}\prod^8_{i=1}3a_i^2 + 4
    \end{align*}
    We note that $\prod^8_{i=1}-a_i^{-2} = (\prod^8_{i=1}a_i)^{-2} = 1^{-2} = 1$, so: 

    $$
        \triangle f = 2^8 \cdot \prod^8_{i=1}3a_i^2 + 4\\
    $$
    We then note that $f(x)$ is an even polynomial as all of its powers are even. Hence, $f(-x) = f(x)$. This means that the negation of each of the roots are also roots. Hence, we can pair each of the roots $a_1, \cdots, a_8$ into $b_1, \cdots, b_4, -b_1, \cdots, -b_4$, which gives us the follow since $(-b_i)^2 = b_i^2$: 

    $$\triangle f = 2^8 \cdot (\prod^4_{i=1}3b_i^2 + 4)^2$$

    We now denote a new polynomial $g(x) = x^4 + x + 1 \in F[x]$. Because $f(x) = (x^2 - b_1^2) \cdots (x^2 - b_4^2)$ and that $f(x) = g(x^2)$, we get that $g(x) = (x - b_1^2) \cdots (x - b_4^2)$. Hence, it splits completely and it has 4 roots.\\

    Hence, let $h(x) = 3^4 \cdot (((x-4)/3)^4 + ((x-4)/3) + 1)$ and $h(x) \in F[x]$. Note that $char(F) \neq 3$, so we can divide by 3. We also note that: 
    \begin{align*}
        h(x) &= 3^4 \cdot ((x-4)/3 - b_1^2)\cdots((x-4)/3 - b_4^2)\\
        &= ((x-4) - 3b_1^2) \cdots ((x-4) - 3b_4^2) \\
        &= (x - (3b_1^2 + 4)) \cdots (x- (3b_4^2 + 4)) 
    \end{align*}
    Thus, $h(x)$ is monic and splits completely. We expand to get $h(x) = (w-4)^4 + 27(w-4) + 81$. We then calculate $h(0) = 256 - 108 + 81 = 229$. We apply Vieta to get that $(-1)^4\prod^4_{i=1}3b_i^2 + 4 = 229$ as they are roots of $h(x)$. Hence, we substitute it in and get that for $charF \neq 3$: 

    $$\triangle f = 2^8 \cdot 229^2$$

    And we are done. \hfill $\square$ \\

    \item 
    To start, we prove the $f(x) = x^8 + x^2 + 1$ is reducible in $\mathbb{F}_p[x]$ for any prime $p \neq 3$. 
    
    By contradiction, $f(x)$ is irreducible in $\mathbb{F}_p[x]$ for some prime $p \neq 3$. Let $p$ be one such prime. Since $f(x)$ is a monic irreducible polynomial in $\mathbb{F}_p[x]$, by Q1(b), we can factor $f(x)$ in $\mathbb{F}_p[x]/(f(x))$ as: 
    
    $$f(x) = \prod_{i = 0}^{8-1} (x- \alpha^{p^i})$$

    We also note that: 

    $$\triangle f = \prod_{0 \leq i < j \leq 7}(\alpha^{p^i} - \alpha^{p^j})^2$$
    
    From (b), since $char(\mathbb{F}_p) \neq 3$, $\triangle f = 2^8 \cdot 229^2$. However, this looks very similar to the $\beta$ defined in HW1(d) where: 

    $$\beta = \prod_{0 \leq i < j \leq 7}(\alpha^{p^i} - \alpha^{p^j})$$

    Hence, $\beta = 2^4 \cdot 229$. Since $2^4 \cdot 229 = 2^4 \cdot 229 \cdot 1$, $\beta \in \mathbb{F}_p$. However, from HW1(d), $\beta \in \mathbb{F}_p$ if and only if $\deg f$ is odd. Clearly, $\deg f$ is not odd, so we reached a contradiction. Thus, $f(x)$ is reducible in $\mathbb{F}_p[x]$ for any prime $p \neq 3$. 
    
    We now consider the case $p = 3$. We note that:
    \begin{align*}
        (x^2-1)(x^6 + x^4 + x^2 + 2) &= (x^8 + x^6 + x^2 + 2x^2) - (x^6 + x^4 + x^2 + 2) \\
        &= x^8 + x^2 - 2\\
        &= x^8 + x^2 + 1
    \end{align*}

    Hence, if $p = 3$, $f(x)$ is reducible in $\mathbb{F}_3[x]$. \hfill $\square$ 

\end{enumerate}


\end{enumerate}

\end{document}