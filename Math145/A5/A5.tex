\documentclass{article}
\usepackage{enumitem}
\usepackage{mathtools,amssymb,amsthm} % imports amsmath
\usepackage{enumitem}
\usepackage[a4paper, total={6in, 9in}]{geometry}
\newtheorem{remark}{Remark}

\begin{document}

\begin{enumerate}

\item

\begin{enumerate}
    \item
    Since $n, o_+(a) \in \mathbb{N}$, we can apply the Division Algorithm to get that $n = q\cdot o_+(a) + r$ for some $q, r \in \mathbb{N}$ where $r < o_+(a)$. However, note that 
    \begin{align*}
      (q\cdot o_+(a) + r)a &= 0  \\
      q\cdot o_+(a)\cdot a + ra &= 0 \\
      ra &= 0 
    \end{align*}
    
    But $r \leq o_+(a)$, which contradicts the minimality of $o_+(a)$ if $r \not = 0$, so it must be that $ r = 0$. Thus, we get that $o_+(a) \mid n$ as desired. \\

    \item 
    If $char(R) = 0$, then $ o_+(a) \mid 0$ if $o_+(a)$ exists. If $char(R) \not = 0$, then it is the smallest positive integer where $char(R) \cdot 1_R = 0$. Thus, for $a \in R$, and from Exercise 7.3 b), we get that 
    \begin{align*}
        char(R) \cdot a &= char(R) \cdot (1_R \cdot a) \\
        &= (char(R) \cdot 1_R) a \\
        &= 0 \cdot a\\
        &= 0
    \end{align*}

    Thus, we get that $char(R) \cdot a = 0$, so from a), it follows that $o_+(a) \mid char(R)$ as desired. \\

    \item 
    We first prove Exercise 7.5's result: \\

    \textbf{Proof for Exercise 7.5} \\

    If $char(R) = m$, then $m \cdot 1_R = 0$. We then note that every element in $\hat{R} = \{ 0, 1_R, 2\cdot 1_R, \cdots , (m-1)\cdot 1_R \}$ is nonzero (except for $0$) and unique. If any other element aside from $0$ is equal to $0$, it would contradict the minimality of $char(R)$. Thus, if there exist $d, e \in \mathbb{Z}$ where $0 \leq d < e \leq (m-1)$ and $d \cdot 1_R = e \cdot 1_R$, we get that $(e - d) \cdot 1_R = 0$ where $0 < (e - d) < char(R)$, which again contradicts the minimality of $char(R)$ and proves uniqueness. \\
    
    Since $\hat{R}$ contains $m$ elements and the order of $R$ is $m$, by the pigeonhole principle, there exists a bijective map $g : R \rightarrow R$ where for $r \in R$, $r \mapsto d\cdot 1_R$ where $0 \leq d \leq (m - 1)$. We now construct a map $f: R \rightarrow \mathbb{Z} / m \mathbb{Z}$ where $r \mapsto [d]_m$ for $r = d \cdot 1_R$. \\
    
    We first prove the map is well-defined. For $r \in R$ where $r = d_1 \cdot 1_R = d_2 \cdot 1_R$ for $d_1, d_2 \in \mathbb{Z}$, we get that $(d_1 - d_2)\cdot 1_R = 0$, which implies $o_+(1) = char(R) = m \mid (d_1 - d_2)$ from (a). Note that $f(d_1 \cdot 1_R) - f(d_2 \cdot 1_R)= [d_1]_m - [d_2]_m = [d_1 - d_2]_m$. Since $m \mid (d_1 - d_2)$, we get that $d_1 - d_2 \equiv 0 \pmod m$, thus $[d_1 - d_2]_m = [0]_m$, so $[d_1]_m = [d_2]_m$ as desired. \\
        
    We now prove it is a ring homomorphism by first showing that:
    \begin{align*}
        f(1) &= f(1 \cdot 1_R) = [1]_m \\
        f(0) &= f(0 \cdot 1_R) = [0]_m
    \end{align*}
    
    We then prove for any $r_1, r_2 \in R$ where for $d_1, d_2 \in \mathbb{Z}$ and $r_1 = d_1 \cdot 1_R$ and $r_2 = d_2 \cdot 1_R$ that: 
    \begin{align*}
        f(r_1 + r_2) &= f((d_1 + d_2) \cdot 1_R) \\
        &= [d_1 + d_2]_m \\
        &= [d_1]_m + [d_2]_m \\
        &= f(r_1) + f(r_2)
    \end{align*}
    \begin{align*}
        f(r_1 \cdot r_2) &= f((d_1 \cdot d_2) \cdot 1_R) \\
        &= [d_1 \cdot d_2]_m \\
        &= [d_1]_m \cdot [d_2]_m \\
        &= f(r_1) \cdot f(r_2)
    \end{align*}
    
    Thus, we get that $f$ is a ring homomorphism. We now prove injectivity. We first assume $r_1, r_2 \in R$ with $r_1 = d_1 \cdot 1_R$ and $r_2 = d_2 \cdot 1_R$ for $d_1, d_2 \in \mathbb{Z}$ where $f(r_1) = f(r_2)$, then $f(r_1 - r_2) = f(0) = 0$. Thus, $[d_1 - d_2]_m = [0]_m$, which implies $d_1 - d_2 \equiv 0 \pmod m$, so $m \mid (d_1 - d_2)$. There exists an integer $q$ where $mq = d_1 - d_2$, so $(d_1 - d_2) \cdot 1_R = q \cdot (m \cdot 1_R) = q \cdot 0 = 0$. Thus, $r_1 - r_2 = 0$ or $r_1 = r_2$. Note that $R$ and $\mathbb{Z} / m \mathbb{Z}$ both have order $m$. By the pigeonhole principle, injective maps between sets of the same size are also surjective. Hence, $f$ is bijective, which gives us $R \cong \mathbb{Z} / m \mathbb{Z}$ as desired. \textbf{(End of proof for Exercise 7.5)}\\

    We now return to c). We first denote $\hat{R} = \{0, a, 2 \cdot a, \cdots , (m-1) \cdot a \}$. Note that each element is non-zero (except for $0$) or else they contradict the minimality of $o_+(a) = m$. If there exist $d, e \in \mathbb{Z}$ where $0 \leq d < e \leq (m-1)$ and $d \cdot a = e \cdot a$, we get that $(e - d) \cdot a = 0$ where $0 < (e - d) < m$, which again contradicts the minimality of $o_+(a)$ and proves uniqueness. \\

    Note that the size of $\hat{R}$ is $m$, which is the order of $R$. By the pigeonhole principle, it implies there exists a bijective map $g: R \rightarrow R$ where for every $r \in R$, $r \mapsto d \cdot a$ for $0 \leq d \leq (m - 1)$ for some $d \in \mathbb{Z}$. Thus, for $a^2 = ka$ and $1_R = ba$ for some integers $k, b$ and $0 \leq k, b \leq m-1$. We get as follows:
    \begin{align*}
    a \cdot 1_R &= ba^2\\
    &= bka \\
    &= a
    \end{align*}
    
    Since $bka = a$, we note that $(bk - 1) a = 0$. By (a), this implies $m \mid (bk - 1)$. This implies there exists an integer $q$ where $bk - 1 = qm$, so $bk - qm = 1$. This implies that $\gcd(m, b) = 1$. Let $t \cdot 1_R = 0$ where $t$ is a non-negative integer and $t \cdot 1_R = tb \cdot a = 0$. This implies that $m \mid tb$. By Corollary 2.20, since $\gcd(m, b) = 1$, we get that $m \mid t$. The smallest positive integer $t$ can be is $m$. Thus, $\operatorname{char}(R) = m$. Thus, we can apply Exercise 7.5 and get $R \cong \mathbb{Z} / m \mathbb{Z}$ as desired. \\


    \item 
    We claim that $o_+(a+b) = mn$. By contradiction, we assume $t$ as a positive integer where $t(a + b) = 0$ and $t < mn$. By the Exercise 7.3 b), note that:
    \begin{align*}
    m(ta + tb) &= 0 \\ 
        t(ma) &= -mtb \\
        0 &= -mtb 
    \end{align*}

    This implies $n \mid mt$, but $\gcd(m, n) = 1$, so by Corollary 2.20, $n \mid t$. If we then do $n(ta + tb)$, it follows that we can conclude $m \mid t$. We then note that since $gcd(m,n) = 1$, we get that $\gcd(m, n) \cdot \text{lcm}(m, n) = \text{lcm}(m, n) = |nm| = nm$. By Exercise 4.1, since $n \mid t$ and $m \mid t$, we get that by Exercise 4.1 that $mn \mid t$. This is a contradiction since $t < mn$. Thus, $o_+(a + b) = mn$. \\ 

    \item 
    By contradiction, we assume $char(R)$ is neither $0$ or a prime. If $char(R) = 1$, then $ 1_R = 0$, which is impossible for a ring. Meanwhile, if $char(R) \not = 1, 0, \text{or any prime } p$, we get that there exist positive integers $m, n$ where $m, n \not = 1, char(R)$ and $mn = char(R)$. Note that since $1 < m, n < char(R)$, $m \cdot 1$ and $n \cdot 1$ are non-zero elements of the ring or it contradicts the minimality of $char(R)$. However, we get that $(m \cdot 1_R) \cdot (n \cdot 1_R) = mn \cdot 1_R = 0$. This is a contradiction to the definition of an integral domain.

\end{enumerate}

\newpage

\item 
\begin{enumerate}
    \item 
    For the reflexive property, we note that: 
    \begin{align*}
        f(x) = f(x + 0) \ \ \forall x \in \mathbb{F}_p         
    \end{align*}

    Thus, we get that $f \sim f$. \\

    For the symmetric property, assume a $g \in S(r)$ where $f \sim g$, we get that there exist an $a \in \mathbb{F}_p$ where
    \begin{align*}
        f(x) &= g(x + a) \ \ \forall x \in \mathbb{F}_p \\
        f(x-a) &= g(x) \\
        g(x) &= f(x + (-a)) 
    \end{align*}

    Thus, we get that $g \sim f$. \\

    For the transitive property, assume $g, h \in S(r)$ where $f \sim g$ and $g \sim h$. We note that there exist an $a, b \in  \mathbb{F}_p $ where $f(x) = g(x+ a)$ and $g(x) = h(x + b)$. Hence: 
    \begin{align*}
        g(x+ a) &= h((x + a) + b) \\
        f(x) &= h((x+a) + b)\\
        f(x) &= h(x + (a  + b))
    \end{align*}

    Thus, we get that $f \sim h$. Since the relation satisfies all the property of an equivalence relation, it is an equivalence relation. \\

    \item 
    Let $f \in S(r)$. We first consider the case where $f$ is a constant function: for all $x \in \mathbb{F}_p$, $f(x) = d$ for some $d \in R$. Thus, for any $g \in S(r)$ with $f \sim g$, we note that its output must also be constant, so $f = g$. Since only $f \sim f$, it follows that the only element in its equivalence class is itself. Thus, the size is $1$. \\
    
    Meanwhile, we consider the case where $f$ is not a constant function. Let $a \in \mathbb{F}_p$, and define $g_a(x) = f(x+a)$. By Lemma 1.1, the map $x \mapsto x + a$ for $x \in \mathbb{F}_p$ is bijective. Thus there do not exist $x_1 \neq x_2$ in $\mathbb{F}_p$ with $x_1 + \hat{a} = x_2 + \hat{a}$, and hence there exists a unique $\hat{x} \in \mathbb{F}_p$ with $g_a(\hat{x}) = f(x)$. Hence, 
    \[
    \sum_{x \in \mathbb{F}_p} f(x) = \sum_{\hat{x} \in \mathbb{F}_p} g_a(\hat{x}) = r.
    \]
    Thus, $g_a \in S(r)$ and $f \sim g_a$. Let $G=\{g_a : a \in \mathbb{F}_p\}$ and note that $f \in G$ because $f = g_0$. For all $g \in G$, they are equivalent to each other since we proved $f \sim g$. Let $a,b \in \mathbb{F}_p$ with $a \neq b$, we prove that $g_a \neq g_b$. By contradiction, suppose $g_a = g_b$. This implies that
    \[
    f(x + a) = f(x + b)\quad \forall x \in \mathbb{F}_p.
    \]
    Denote $c = b - a$ and we get that: 
    \begin{align*}
    f(x + a - b + c) &= f(x + b - b + c),\\
    f(x) &= f(x + c).
    \end{align*}
    We also have $f(x + c) = f(x + c + c)$, so by repeating this $k$ times for any positive integer $k$ we get $f(x) = f(x + kc)$. Since $a \neq b$, we have $c \neq 0$. We now show that $x, x + c, x + 2c, \dots, x + (p-1)c$ are all distinct. By contradiction, there exist integers $i \neq j$ with $0 \le i < j \le p - 1$ such that $x + ic = x + jc \ \ \text{in } \mathbb{F}_p$.

    Then $jc - ic = (j - i)c = 0$ in $\mathbb{F}_p$, i.e., $(j-i)c \equiv 0 \pmod{p}$. However, $c \neq 0$ in $\mathbb{F}_p$, so $c \not\equiv 0 \pmod{p}$. Thus, by Euclid's Lemma, $p \mid (j - i)$, which is a contradiction since $0 < (j - i) \le p - 1$. \\
    
    Since $x, x + c, x + 2c, \dots, x + (p-1)c$ are $p$ distinct elements of $\mathbb{F}_p$ and $|\mathbb{F}_p| = p$, they represent every element of $\mathbb{F}_p$. Hence $f(x) = f(x + kc)$ for all $k$ implies that $f$ is constant, a contradiction. Thus, $g_a \neq g_b$. Since there are $p$ distinct elements in $\mathbb{F}_p$, there are also $p$ distinct functions in $G$. Thus, the equivalence class size is $p$.

\end{enumerate}

\newpage 

\item 
Let $I$ be an ideal of $R$. If $I=\{0\}$, then $I=(0)$, which is a principal ideal. If $I\neq\{0\}$, let the set $P=\{N(x): x\in I,\ x\neq 0\}$. Since $P$ is nonempty as $I\neq\{0\}$ and $P\subset \mathbb{Z}$, by the well-ordering principle, there exists a smallest element. We denote $d\in I$ where $N(d)$ is the smallest element of $P$. We note that $(d)\subseteq I$. Let $n\in I$, since $R$ is a Euclidean domain, there exist $q,r\in R$ where $n=dq+r$. We note that $r=n-dq\in I$ since $n,d\in I$. However, we note that $N(r)<N(d)$ if $r$ is nonzero, which contradicts the minimality of $N(d)\in P$. Hence, it must be that $r=0$, so $d\mid n$ and $n\in(d)$. This proves $I\subseteq(d)$, which implies $I=(d)$. Hence, every ideal $I\in R$ is a principal ideal, so $R$ is a PID.


\newpage

\item 
\begin{enumerate}
    \item 
    We first note that $(0, 1)$ is not a unit because if it is, there exist $(a, b) \in \mathbb{Z}^2$ with $(0, 1)\cdot(a,b)=(1,1)$. Then $0 \cdot a = 1$, which is impossible for any given $a \in \mathbb{Z}$. Thus, since $(0, 1) = (0, 1) \cdot (0,1)$, it can be expressed as the product of non-units, so it is not irreducible. \\

    Meanwhile, let $(a, b), (p, q) \in \mathbb{Z}^2$. If $a = 0$, then we set $q = b$, and we get that $(0, 1) \cdot (p, q) = (0, b)$. Thus, $(0, 1) \mid (a,b)$. Hence, we also note the contrapositive that if $(0, 1) \nmid (a, b)$, then $a \neq 0$. \\
    
    By contradiction, $(0, 1)$ is not prime, so there exist $(a, b), (c, d) \in \mathbb{Z}^2$ where $(0, 1) \mid (ac, bd)$ but $(0, 1) \nmid (a, b)$ and $(0, 1) \nmid (c, d)$. This implies both $a, c \neq 0$, and since $\mathbb{Z}$ is an integral domain, $ac \neq 0$. However, since  $(0, 1) \mid (ac, bd)$, it is also implied that $ac = 0$, which is a contradiction. Thus, $(0, 1)$ is prime. \\


    \item 
    By Lemma 9.2, for any $f, g \in R$, we get that $\deg(fg) = \deg(f) + \deg(g)$. By contradiction, $x$ is not irreducible, so there exist two non-unit polynomials $f, g$ where $fg = x$. However, from the lemma, we note that since the degree of polynomials are non-negative integers, either $\deg(f) = 1$ and $\deg(g) = 0$ or vice versa. Assume $\deg(f) = 0$, which implies $f \in \mathbb{Q}$ as it is a constant. However, since $f \in \mathbb{Q}$ and $f \neq 0$ and $\mathbb{Q}$ is a field, $f^{-1} \in \mathbb{Q} \subset R$. Thus, $f$ is a unit, which is a contradiction. Thus, $x$ is irreducible. \\

    Meanwhile, note that $(\sqrt{2}x) \cdot (\sqrt{2}x) = 2x^2 = 2x \cdot x$. Thus, $x \mid (\sqrt{2}x) \cdot (\sqrt{2}x)$. However, we note that if $x \mid (\sqrt{2}x)$, then there exists a $q \in R$ where $xq = \sqrt{2}x$. $R$ is an integral domain and since $x(q - \sqrt{2}) = 0$, it implies $q = \sqrt{2}$. This is a contradiction because $\sqrt{2} \notin R$. Thus, $x \nmid (\sqrt{2}x)$, which proves $x$ is not prime. \\

    \item 
    Assume $r$ is prime, then if $r \mid ab$, then $r \mid a$ or $r \mid b$. By contradiction, assume $r$ is not irreducible, so $r = de$ where $d, e \in R$ and are non-units. Since $r = de$, we assume $r \mid d$, thus there exist $d \in R$ where $d = rq$. Hence, we get that $r = rqe$. Since $R$ is a PID, it is an integral domain, thus: 
    \begin{align*}
        r &= rqe \\
        0 &= r(qe -1) \\
        qe &= 1
    \end{align*}

    This implies $e$ is a unit, which is a contradiction! Thus, if $r$ is prime, then $r$ is irreducible. \\

    For the converse, assume $r$ is irreducible. By contradiction, $r$ is not prime, so there exist a $r \mid ab$, but $r \nmid a$ and $r \nmid b$. We then denote the ideal $(r, a) = \{rx + ay: x, y \in R \}$. Since $R$ is a PID, there exist a principal ideal $(d)$ for $d \in R$ where $(r, a) = (d)$. Since $r \in (r, a)$, we note that $d \mid r$ and $d \mid a$. \\

    Since $d \mid r$, there exists a $u \in R$ where $du = r$. If $d$ is a non-unit, we note that $u$ must be a unit or it contradicts that $r$ is irreducible. However, this implies that $u^{-1}r = d \cdot 1$. Since $d \mid a$, there exist an $k \in R$ where $a = dk = (u^{-1}r)k$. This implies that $r \mid a$, which is a contradiction. Thus, $d$ is a unit, so $(d) = (1)$ by Lemma 8.5. \\

    Since $(r, a) = (1)$, we get that $1 \in (r,a)$, so there exist $x, y \in R$ where $rx + ay = 1$. Thus, if we multiply both sides by $b$, we get that $brx + bay = b$. Since $r \mid ab$, so there exist a $z \in R$ where $rz = ab$. We get that $r(bx + zy) = b$. This implies $r\mid b$, which is a contradiction to $r$ being not prime. Hence, if $r$ is irreducible, then $r$ is prime. 
\end{enumerate}




\end{enumerate}

\end{document}