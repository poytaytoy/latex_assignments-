\documentclass{article}
\usepackage{enumitem}
\usepackage{mathtools,amssymb,amsthm} % imports amsmath
\usepackage{enumitem}
\usepackage[a4paper, total={6in, 9in}]{geometry}
\newtheorem{remark}{Remark}

\begin{document}

\begin{center}
    \textbf{\large Greatest Common Divisor}
\end{center}

\begin{enumerate}
    \item
    \textbf{Proposition 2.15} Let $a, b, q$ be integers. Then $\gcd(a, b) = \gcd(a, b - aq)$ 

    \item
    \textbf{Proposition 2.19} Let $a, b, c$ be integers such that $\gcd(a, c) = 1$. Then $\gcd(c, ab) = \gcd(c, b)$.    

    \item
    \textbf{Corollary 2.20} Let $a, b, c$ be integers. Suppose $c \mid ab$ and $\gcd(a, c) = 1$. Then $c \mid b$ 

    \item 
    \textbf{Euclid's Lemma} (Pls remember it exists) If $p$ prime then if $p \mid ab$ then $p \mid a$ or $p \mid b$.  
\end{enumerate}

\newpage

\begin{center}
    \textbf{\large Distribution of Primes}
\end{center}


\begin{enumerate}

\item 
List of prime numbers 
\begin{table}[h]
    \centering
    \begin{tabular}{|*{10}{c|}}
    \hline
    2 & 3 & 5 & 7 & 11 & 13 & 17 & 19 & 23 & 29 \\ \hline
    31 & 37 & 41 & 43 & 47 & 53 & 59 & 61 & 67 & 71 \\ \hline
    73 & 79 & 83 & 89 & 97 & 101 & 103 & 107 & 109 & 113 \\ \hline
    127 & 131 & 137 & 139 & 149 & 151 & 157 & 163 & 167 & 173 \\ \hline
    179 & 181 & 191 & 193 & 197 & 199 & 211 & 223 & 227 & 229 \\ \hline
    233 & 239 & 241 & 251 & 257 & 263 & 269 & 271 & 277 & 281 \\ \hline
    283 & 293 & 307 & 311 & 313 & 317 & 331 & 337 & 347 & 349 \\ \hline
    353 & 359 & 367 & 373 & 379 & 383 & 389 & 397 & 401 & 409 \\ \hline
    419 & 421 & 431 & 433 & 439 & 443 & 449 & 457 & 461 & 463 \\ \hline
    467 & 479 & 487 & 491 & 499 & 503 & 509 & 521 & 523 & 541 \\ \hline
    547 & 557 & 563 & 569 & 571 & 577 & 587 & 593 & 599 & 601 \\ \hline
    607 & 613 & 617 & 619 & 631 & 641 & 643 & 647 & 653 & 659 \\ \hline
    661 & 673 & 677 & 683 & 691 & 701 & 709 & 719 & 727 & 733 \\ \hline
    739 & 743 & 751 & 757 & 761 & 769 & 773 & 787 & 797 & 809 \\ \hline
    811 & 821 & 823 & 827 & 829 & 839 & 853 & 857 & 859 & 863 \\ \hline
    877 & 881 & 883 & 887 & 907 & 911 & 919 & 929 & 937 & 941 \\ \hline
    947 & 953 & 967 & 971 & 977 & 983 & 991 & 997 & & \\ \hline
    \end{tabular}
    \caption{Primes from 1 to 1000}
\end{table}

\item 
Key Identity $\rightarrow$ $2^n \leq L_n \leq 4^{n-1}$ \\

\item 
\textbf{Legrende Formula}: $$\nu_p(n!) = \sum^{\infty}_{k = 1} \lfloor \frac{n}{p^k} \rfloor = \sum^{\lfloor \log_p{n} \rfloor}_{k = 1} \lfloor \frac{n}{p^k} \rfloor = \lfloor \frac{n}{p} \rfloor + \lfloor \frac{n}{p^2} \rfloor + \cdots$$ 

\item 
\textbf{Lemma 4.5} Let $p$ be a prime and let $n = 1, \cdots, p - 1$. Then $\nu_p(\binom{p}{n}) = 1$

$$\nu_p(\binom{p}{n}) = \lfloor \frac{p}{p} \rfloor - \lfloor \frac{n}{p} \rfloor - \lfloor \frac{p - r}{p} \rfloor = 1 - 0 - 0 = 1$$

\item 
$$\lfloor \frac{n}{a} \rfloor - \lfloor \frac{m}{a} \rfloor - \lfloor \frac{n - m}{a} \rfloor = 
\begin{cases}
    1 & \text{if $n \% a \leq m \% a$ } \\  
    0 & \text{if $n \% a \geq m \% a$ }   
\end{cases}$$  

\item
\textbf{LTE (p is odd)} If $p \nmid a$ and $p \nmid b$ with $p \mid a - b$. Then: 

$$\nu_p(a^n - b^n) = \nu_p(a + b) + \nu_p(n)$$

\item
\textbf{LTE (p is even)} If $a, b$ are odd and $n$ is even, then: 

$$\nu_2(a^n - b^n) = \nu_2(a^2 + b^2) + \nu_2(n) - 1$$

\item 
Cool technique I learned while practicing: 

$$10! = \frac{\frac{\frac{10!}{5! \cdot 5!}}{
\frac{5!}{2! \cdot 3!} \cdot \frac{5!}{2! \cdot 3!}
}}{2! \cdot 3! \cdot 2! \cdot 3!}$$
\end{enumerate} 

\newpage 

\begin{center}
    \textbf{\large Cyclotomic Polynomials}
\end{center}

\begin{enumerate}
    \item 
    $\Phi_1(x) = x - 1 \\
    \Phi_2(x) = x + 1 \\
    \Phi_3(x) = x^2 + x + 1 \\
    \Phi_4(x) = x^2 + 1 \\
    \Phi_5(x) = x^4 + x^3 + x^2 + x + 1 \\
    \Phi_6(x) = x^2 - x + 1 \\
    \Phi_7(x) = x^6 + x^5 + x^4 + x^3 + x^2 + x + 1 \\
    \Phi_8(x) = x^4 + 1 \\
    \Phi_9(x) = x^6 + x^3 + 1 \\
    \Phi_{10}(x) = x^4 - x^3 + x^2 - x + 1 \\
    \Phi_{11}(x) = \text{too long but you should know} \\
    \Phi_{12}(x) = x^4 - x^2 + 1$

    \item 
    \textbf{Exercise 5.1} Euler Trotient $\phi$ properties: 

    \begin{enumerate}
        \item 
        $\phi(1) = 1$, $\phi(p) = p - 1$ for prime $p$ 

        \item 
        $\phi(m) = 2$ iff $m = 3,4,6$. 

        \item 
        $\phi(p^k) = p^k - p^{k - 1}$

        \item 
        $\phi(mn) = \phi(m) \cdot \phi(n)$ if $m$ and $n$ are co-prime integers
    \end{enumerate}

    \item 
    $x^m - 1 = \prod_{d \mid m} \Phi_d(x) \rightarrow \text{If p prime, } \Phi_p(x) = \frac{x^p - 1}{x-1}$ \\

    \item 
    HW4Q1d

    \begin{enumerate}
        \item   
        Let $p$ be prime. If $p \nmid \Phi_p(a)$, then $\Phi_p(a) \equiv 1 \pmod p$. 

        \item 
        If $m \geq 2$
        $$\Phi_m(1) = \begin{cases}
            q & \text{if $m = q^k$ for some prime $p$} \\
            1 & \text{otherwise}  
        \end{cases}$$
    \end{enumerate}

    

    \item 
    \textbf{Proposition 6.8 } Let $m \in \mathbb{N}$ and $n > 1$ coprime to $m$. Then $n \mid \Phi_m(a) \implies o_n(a) =m$

    \item 
    \textbf{Corollary 6.9 } Let $p$ be prime. If $p \mid \Phi_m(a)$, then $p \mid m$ or $p \equiv 1 \pmod m$ (because FLT forces it that $m \mid p - 1$). 
    
    \item
    For $m \geq 1$, then $\Phi_m(a) \equiv 1 \pmod a$. (Because all cyclotonic polynomials end with a constant 1) 
    
    \item 
    Let $p$ be prime, if $p \mid \Phi_p(a)$, then $\nu_p(\Phi_p(a)) = 1 \implies \nu(\Phi_p(a)) \leq 1$ for any $a$.
    
    \item 
    Let $m$ be odd. We get that $\Phi_m(-x) = \Phi_{2m}(x)$. 


\end{enumerate}

\newpage

\begin{center}
    \textbf{\large Abstract Algebra}
\end{center}

\begin{enumerate}

    \item 
    In case, you met a problem for some reason on additive order in a ring; pls check HW5 Q1

    \item
    A non-zero non-unit element $r \in R$ is said to be 
    \begin{enumerate}
        \item 
        \textbf{irreducible} if it cannot be written as $r = ab$ where $a$ and $b$ are not units 
        \item
        \textbf{prime} if whenever $r \mid ab$ in $R$ for some $a, b \in R$, we have $r \mid a$ or $r \mid b$ 
    \end{enumerate}

    \item 
    \textbf{CRT}: Let $m, n \geq 2$ be co-prime integers. Then 

    $$\mathbb{Z}/mn\mathbb{Z} \cong \mathbb{Z}/m\mathbb{Z} \times \mathbb{Z}/n\mathbb{Z} $$
    
    \item 
    An ideal $I$ of $R$ satisfies these properties: 
    \begin{enumerate}
        \item
        $0 \in I$ 
        \item 
        for any $a, b \in I$, we have $a + b \in I$ 
        \item 
        for any $a \in I$, and any $r \in R$, we have $ra \in I$
    \end{enumerate}

    \item 
    \textbf{Lemma 8.5} If an ideal $I$ of $R$ contains a unit, then $I = R$.
    
    \item 
    \textbf{Corollary 8.14} If $R$ is a field, then any $\varphi: R \rightarrow S$ is injective. 


    %TODO Look into FIT a little bit to make you know how it works 

    \item 
    \textbf{Theorem 8.19 (First Isomorphim Theorem)} Let $\varphi : R \rightarrow S$ be a ring homomorphism. Then the natural map $\psi: R/\ker(\varphi) \rightarrow im(\varphi)$ sending the coset $[a]$ to $\varphi(a)$ is an isomorphism.  
    \item 
    HW6 3
    \begin{enumerate}
        \item 
        A proper ideal $I$ is a prime ideal if for any $a, b \in R$, if $I \mid ab$, then $I \mid a$ or $I \mid b$. 
        \item 
        $I$ is a prime ideal of $R$ iff $R/I$ is an integral domain. 
        \item 
        Let $S$ be a subring of $R$ and let $I$ be a prime ideal of $R$. $S \cap I$ is a prime ideal of $S$.
         
    \end{enumerate}

    \item 
    \textbf{HW7 Q3} Let $F$ be a field of char $p$ and let $m$ be where $p \nmid m$. 

    \begin{enumerate}
        \item 
        if $a \in F^{\times}$ satisfies $\Phi_m(a) = 0$ then $o(a) = m$ 
        \item 
        if $a \in F^{\times}$ satisfies $o(a) =m$, then $\Phi_m(a) = 0$ and $\Phi'_m(a) = 0$. 
    \end{enumerate}
    

\end{enumerate}

\newpage 

\begin{center}
    \textbf{\large Polynomial Ring}
\end{center}

\begin{enumerate}
    \item 
    \textbf{Division Algorithmn for Polynomials} Let $R$ be a com ring. Let $f(x) \in R[x]$ and let $g(x) \in R[x]$. If the leading coefficient is the a unit in $R$ then $\exists q(x), r(x) \in R[x]$ s.t. 

    $$f(x) = g(x)q(x) + r(x) \ \text{and} \ \deg(r) < \deg(g)$$

    \item 
    \textbf{Props 9.5} A homomorphism $R[x] \rightarrow S$ is the same as $R[x] \rightarrow R \rightarrow S$ with $R[x] \rightarrow R$ being an evaluation homorphism of $f(\alpha)$ for $f(x) \in R[x]$. \\

    \item 
    \textbf{Corollary 9.7} Let $R$ be an integral domain. $f(x) \in R[x]$ and $c_1, \cdots, c_n$ be distinct then they are roots iff $(x - c_1)\cdots(x - c_n) \mid f(x)$. 

    \item 
    \textbf{Lemma 9.10 (Fropbenius Map)} Let $R$ be a com ring with $char p$. Then the map $r \mapsto r^p$ is a ring homomorphism. In other words, 

    $$(a + b)^{p^n} = a^{p^n} + b^{p^n}$$

    \item 
    \textbf{Proposition 9.12} $c$ is a repeated root of $f(x)$ iff $f(c) = f'(c) = 0$. 
    
    \item 
    \textbf{HW8 Q3} If $f(x) \in F[x]$ is a monic polynomial so that $f(x) = (x-a_1)\cdots(x-a_n)$ with roots in $F$ (it splits completely). Then its discrimnant is defined as: 

    $$\Delta (f) = \prod_{1 \leq i \leq j \leq n} (a_i - a_j)^2$$

\end{enumerate}

\newpage 

\begin{center}
    \textbf{\large Finite Fields}
\end{center}

\begin{enumerate}
    \item 
    \textbf{Props 9.16} If $F$ is a finite field and let $g(x) \in F[x]$ with $\deg \geq 1$. Then, the following are equiv: 

    \begin{enumerate}
        \item 
        $g(x)$ is irreducible
        \item 
        $F[x]/(g(x))$ is a field 
        \item 
        $F[x] / (g(x))$ is an integral domain
    \end{enumerate}

    \item 
    \textbf{Lemma 9.18} Let $F$ be a field. Let $g(x) \in F[x]$ be a polynomial of $\deg \geq 1$ then $|F[x]/g(x)| = |F|^d$. 

    \item 
    \textbf{Props 9.18} Let $g(x) \in \mathbb{F}_p[x]$ be a polynomial. Let $R$ be a com ring with char $p$. Then $\mathbb{F}_p[x]/g(x) \rightarrow R$ is the same as giving a root $\beta$ of $g(x)$ in $R$ 

    \item 
    \textbf{Theorem 9.23} Let $p$ be prime and let $n \in \mathbb{N}$. Then $x^{p^n} -x$ is the product of all monic irreducible polynomials in $\mathbb{F}_p[x]$ of degree $d$ where $d \mid n$  

    \item 
    \textbf{From Proof of Theorem 9.23} Let $S_p(n)$ denote he set of monic irreducible polynomials in $\mathbb{F}_p[x]$ of degree $n$. Then 

    $$n \cdot |S_p(n)| = \sum_{d \mid m} \mu(d) p^{n/d}$$

    A thing to note is that $p \mid |S_p(n)|$. 
    \item 
    \textbf{Theorem 10.1}
    \begin{enumerate}
        \item 
        Every finite field has order $p^d$ for some $p$ and positive integer $d$. (basically also imples $\mathbb{F}_p^d$ has char p)
        \item 
        Any two finite fields of the same order are isomorphic 
        \item 
        For every prime $p$ and every positive int $d$, there is a $g(x) \in \mathbb{F}_p[x]$ irreducible of degree $d$. In other words, $\mathbb{F}_{p^d} \cong \mathbb{F}_p[x]/(g(x))$ 
        \item 
        There exists a homomorphism $\mathbb{F}_{p_1^{d_1}} \rightarrow \mathbb{F}_{p_2^{d_2}}$ iff $p_1 = p_2$ and $d_1 \mid d_2$
    \end{enumerate}

    \item 
    \textbf{Theorem 10.7 (Existence of primitive element)} Let $F$ be a finite field, then there is a $\alpha \in F^{\times}$ where $o(a) = |F| - 1$. 
    
    \item 
    Result from Proof of Theoreom 10.7 $\rightarrow$ Let $F$ be a field of $q$ elements. For $d \mid q - 1$, let $N_d$ denote the number of elements in $F$ with order $d$. $N_d = \phi(d)$.

    \item 
    \textbf{Corollary 10.10} Let $F$ be a field of order $p^n$ and positive int $n$. Then $\alpha \in F$ is in $\mathbb{F}_p$ iff $\alpha^p = \alpha$. (Because $x^p -x$ has all the roots in $\mathbb{F}_p$) 

    \item 
    \textbf{Example 10.11} If $a \in \mathbb{F}_{p^2}$, we have $a^{p + 1} \in \mathbb{F}_{p}$ because $a^{p^2}\cdot a^p = (a^p\cdot a)^p$ and $a^{p^2} = a$. 

    \item
    \textbf{Lemma 10.14} Let $p$ be a prime, not dividing $m \in \mathbb{N}$. Let $\alpha \in \mathbb{F}^{\times}_{p^2}$ with $o(\alpha) = m$. Then $\alpha + \alpha^{-1} \in \mathbb{F}^{\times}_p$ iff $p \equiv \pm 1 \pmod m$.  
    

\end{enumerate}

\newpage 
\begin{center}
    \textbf{\large Quadratic Reciprocity}
\end{center}

\begin{enumerate}
    \item   
    \textbf{Theoreom 11.4 (Quadratic Reciprocity)} 

    \begin{enumerate}

        \item 
        Let $p$ be an odd prime for $(a \in \mathbb{F}_p)^{\times}$. Then: 

        $$a^{(p-1)/2} = \begin{cases}
            1 \pmod p & \text{if there is an integer x such that $x^2 \equiv a \pmod p$ }\\
            -1 \pmod p & \text{if there is no such integer }
        \end{cases}$$

        \item 
        $-1$ is a quadratic residue $\pmod p$ iff $p = 2$ or $p \equiv 1 \pmod 4$. In other words, for $p \neq 2$, 

        $$(\frac{-1}{p}) = (-1)^{(p^2 - 1)/2}$$

        \item 
        $2$ is a quadratic residue mod $p$ iff $p = 2$ or $p \equiv \pm 1 \pmod 8$. In other words, for $p \neq 2$, 

        $$(\frac{2}{p}) = (-1)^{(p^2-1)/8}$$
        
        \item 
        If $p, q$ are distinct odd primes, then 

        $$(\frac{p}{q}) = -(\frac{q}{p}) \ \text{if both p,q $\equiv 3 \pmod 4$}$$
        $$(\frac{p}{q}) = -(\frac{q}{p}) \ \text{otherwise}$$
    \end{enumerate}

    \item 
    \textbf{Corollary 11.8} We have 

    \begin{enumerate}
        \item 
        $3$ is a quadratic residue $mod p$ iff $p = 3$ or $p \equiv \pm 1 \pmod {12}$

        \item 
        (Extra from HW9 Q1) $-3$ is a quadratic resdue $mod p$ iff $p = 2$ or $p = 3$ or $p \equiv 1 \pmod 3$. 
        \item 
        $5$ is a quadratic residue $mod p$ iff $p = 5$ or $p \equiv \pm 1, \pm 9 \pmod {20}$
        \item 
        $7$ is a quadratic residue $mod p$ iff $p = 7$ or $p \equiv \pm 1, \pm 9, \pm 25 \pmod {28}$
    \end{enumerate}

    \item 
    \textbf{Theorem 11.13 (Hensel's Lemma)} Let $p$ be a prime. Let $f(x) \in \mathbb{Z}_p[x]$ and $r \in \mathbb{Z}_p$. Suppose 
    $$\nu_p(f(r)) > 2\nu_p(f'(r))$$

    Then there exists $\alpha \in \mathbb{Z}_p$ such that $f(\alpha) = 0$ and $\nu_p(a - r) > \nu_p(f(r)) -\nu_p(f'(r))$. 

    \item 
    \textbf{Corollary 11.16 (Nooby Hensel's Lemma)} Suppose $f(x) \in \mathbb{Z}[x]$. Let $p$ be a prime and let $r \in \mathbb{F}_p$. Suppose $f(r) = 0$ and $f'(r) \neq 0$ in $\mathbb{F}_p$. Then there exists $\alpha \in \mathbb{Z}_p$ such that $f(\alpha) = 0$. 

    \item 
    \textbf{Jacobi Symbol Def} (Just there in case) Given two coprime integers $a,b$ where $b$ is a positive odd integer, we factor $b = p_1 \cdots p_r$ into a product of (possibly equal) odd primes. Then we define the Jacobi symbol
    
    $$
    \left(\frac{a}{b}\right) := \prod_{j=1}^{r} \left(\frac{a}{p_j}\right).
    $$

    \item
    \textbf{Theorem 11.20 (Quadratic Reciprocity for Jacobi Symbols)} Let $b$ be a positive odd integer. Then: 

    \begin{enumerate}
        \item 
        $(\frac{-1}{b}) = (-1)^{(b-1)/2}$
        \item 
        $(\frac{2}{b}) = (-1)^{(b^2 -1)/8}$
        \item 
        if $a$ is a positive odd integer co-prime to $b$, then $(\frac{a}{b})(\frac{b}{a}) = (-1)^{(a-1)(b-1)/4}$
    \end{enumerate}

\end{enumerate}

\newpage 
\begin{center}
    \textbf{\large Group Theory}
\end{center}

\begin{enumerate}
        \item 
        \textbf{Theorem 12.3 (Lagrange's Theorem)} Suppose $H$ is a subgroup of a finite group of $G$. Then $|H| \mid |G|$. 

        \item 
        \textbf{Corollary 12.5} Suppose $G$ is a finte group and $H$ is a proper subgroup. Then $|H| \leq |G|/2$. 

        \item 
        \textbf{Lemma 12.7} Let $m, n \in \mathbb{N}$. Then $C_m \times C_n$ is cyclic iff $\gcd(m,n) = 1$. 

        \item
        \textbf{Theorem 12.8} For any positive integer $t$, 

        $$(\mathbb{Z}/p^t\mathbb{Z})^{\times} \cong C_{p^{t-1}(p-1)} \text{\ if p is odd,}$$

        $$(\mathbb{Z}/2^t\mathbb{Z})^{\times} \cong \begin{cases}
            1 & if \ t = 1 \\
            C_2 & if \ t = 2 \\
            C_2 \times C_{2^{t-2}} & if \ t \geq 3
        \end{cases}$$

        \item 
        \textbf{Corollary 12.10} Let $m \in \mathbb{N}$. Then $(\mathbb{Z}/m\mathbb{Z})^{\times}$ is cyclic iff $m = 2, 4, p^t, 2p^t$. for some odd prime $p$ and positive integer $t$. In particular, if $m$ is an odd composite integer, then $(\mathbb{Z}/m\mathbb{Z})^{\times}$ is not cyclic. 
        
        \item
        \textbf{HW10 Q1}
        $G$ is cyclic with order $m$ iff for any $d \mid m$, there is a unique subgroup of $G$ of order $d$. There are also $\phi(d)$ elements of $G$, assuming it is cyclic, of order $d$. 
    \end{enumerate}

\newpage

\begin{center}
    \textbf{\large Probablistic Primality Testing}
\end{center}

\begin{enumerate}
    \item 
    \textbf{Lemma 13.1} If $a^{n-1} \equiv \pmod n$ for every $a = 1 \cdots n - 1$, then $n$ is prime (Because it implies it is co-prime to every number before for it)

    \item 
    \textbf{Lemma 13.2} Let $F_n = \{ a \in (\mathbb{Z}/n\mathbb{Z})^{\times}: a^{n- 1} = 1\}$. It is a subgroup. (Side Note: if $F_n = (\mathbb{Z}/n\mathbb{Z})^{\times}$ then $n$ is Carmichael and there exists infinitely many of them) 

    \item 
    \textbf{HW10 3(b)} Let $n$ be an odd int wit prime factorization $n = p_1^{k_1} + \cdots + p_r^{k_r}$ where $p_1, \cdots, p_r$ are odd primes. $n$ is Carmichael iff $p_i^{k_i - 1}(p_i - 1)$ divides $n - 1$ for every $i = 1,\cdots, r$. 

    \item 
    \textbf{Props 13.4} Carmichael Numbers are squarefree. (divisible by no-square integer aside from $1$)

    \item 
    Carmichael Numbers $n$ exhibit the property where for all $a \in \mathbb{Z}/n\mathbb{Z}$, we get that $a^n \equiv a \pmod n$. 
    %TODO Investigate EX 13.1 
\end{enumerate}

\begin{center}
    \textbf{\large Deterministic Primality Testing}
\end{center}

\begin{enumerate}
    \item 
    \textbf{Exercise 14.1} Suppose $p \equiv 3 \pmod 4$ is a Sophie Germain prime. We get that $2p + 1 \mid M_p$ and so $M_p$ is not prime for $p > 3$. 

    \item 
    \textbf{Exercise 14.2} Suppose $p, q$ are primes such that $q \mid M_p$. We get that $q \equiv 1 \pmod {2p}$ 
\end{enumerate}

\newpage 

\begin{center}
    \textbf{\large The Gaussian and Eisentein integers}
\end{center}

 Actually fuck it, go to page 101 of package 4 instead for references in this section. 

\begin{enumerate}
    \item 
    \textbf{Theorem 15.1} If $p$ is a prime congruent to $1 \ mod \ 4$, then there exist $a, b \in \mathbb{Z}$ such that $p = a^2 + b^2$ 

    \textbf{Theorem 15.2 } If $p$ is a prime congruent to $1 \ mod \ 3$, then there exist $a, b \in \mathbb{Z}$ such that $p = a^2 -ab + b^2$.
    
    
    
\end{enumerate}


\end{document}