\documentclass{article}
\usepackage{mathtools,amssymb,amsthm} % imports amsmath
\usepackage{enumitem}
\usepackage[a4paper, total={6in, 9in}]{geometry}
\begin{document}



\begin{enumerate}
\item  

\textbf{Lemma:} By the Division Algorithm, for all integers $a$, there exist integers $r$ and $q$ where $a = p\cdot q + r$ and $0 \leq r < p$. Thus, it follows that $a \equiv r \pmod{p}$. We claim that if $r = 1$, then $p \mid \Phi_p(a)$. If $r \neq 1$, then $p \mid \Phi_p(a) - 1$. \\

\textbf{Proof:} 

For $r = 0$ and since $p$ is a prime, we note that:

$$\Phi_p(a) - 1  = \frac{a^p-1}{a-1} - 1= (a^{p-1} + a^{p-2} + \cdots + a^{1} + 1) - 1 = a^{p-1} + a^{p-2} + \cdots + a^{1}$$

It follows that since each term of the sum is in the form $a^k$ for $1 \leq k \leq p - 1$, that $p\mid a^k$ since $p \mid a$, so $p \mid \Phi_p(a) - 1$.\\

For $r = 1$, we note that, since $p$ is a prime: 

$$\Phi_p(a) = \frac{a^p-1}{a-1} = a^{p-1} + a^{p-2} + \cdots + a^{1} + 1$$

There are $p$ terms in the sum. For $p - 1$ of them, they come in the form $a^k$ for $1 \leq k \leq p - 1$ and $a^k \equiv 1 \pmod{p}$ since $a \equiv 1 \pmod{p}$. Since $1 \equiv 1 \pmod{p}$, the congruence of summing all $p$ terms of the sum is $\Phi_p(a) \equiv p \pmod{p}$. Thus, since $p \mid p$, combining the two gives $p \mid \Phi_p(a)$. \\

For $r \neq  0, 1$, we note from Fermat's Little Theorem that $a^{p-1} \equiv 1 \pmod{p}$ as $p \nmid a$, then $\gcd(p, a) = 1$. We then note that:

$$\Phi_p(a) - 1 = \frac{a^p -1 }{a -1} - 1 = \frac{a^p -a}{a -1} = \frac{a(a^{p - 1} - 1)}{a-1}$$

Since $p \nmid a$ and $p \nmid a - 1$, we get that: 

$$\nu_p(\Phi_p(a) - 1) = \nu_p(a(a^{p-1} - 1)) - \nu_p(a-1) = \nu_p(a^{p-1} - 1) $$

We also note that since $\Phi_p(a) - 1$ can be expressed as a sum of powers of $a$ up to $p-1$ as shown when proving $r = 0$, we note that $\Phi_p(a) - 1$ must be an integer. Since $p \mid a^{p - 1} - 1$, we get that $\nu_p(a^{p-1} - 1) \geq 1$, so $\nu_p(\Phi_p(a) - 1) \geq 1$, thus $p \mid \Phi_p(a) - 1$. \\
\begin{enumerate}
    \item 
    From the Lemma, there are only two possible cases for any given integer $a$ and $\Phi_p(a)$, which is either $ p \mid \Phi_p(a) - 1$ or $ p \mid \Phi_p(a) $. Thus, if $ p \nmid \Phi_p(a) $, then the only other possible case is that $ p \mid \Phi_p(a) - 1$. \\

    \item 
    We note that from applying the Division Algorithm to $a$ as done in the Lemma, the remainder $r$ of $a$ is either $1$ or not $1$. If $r \neq 1$, then it follows that $p \mid \Phi_p(a) - 1$, so $p \nmid \Phi_p(a)$. Thus, only when $r = 1$, we get that $p \mid \Phi_p(a)$. Hence, if $p \mid \Phi_p(a)$, we get that $p \mid a - 1$. Since $p \mid a - 1$ and $p > 2$, we apply LTE to get that: 

    \begin{align*}
        \nu_p(\Phi_p(a)) = \nu_p\left(\frac{a^p - 1}{a - 1}\right) &= \nu_p(a^p - 1) - \nu_p(a-1)\\
        &= \nu_p(a - 1) + \nu_p(p) - \nu_p(a-1)\\
        &= 1
    \end{align*}    

    as desired. \\

    \item 
    For $m = 2$, we note that $1$ is the only divisor $d$ where $\gcd(2, d) = 1$. Thus:
    \begin{align*}
        x^{\phi(2)}\Phi_2(1/x) &= x \cdot (1/x - \zeta_2) \\
        &= 1 - x\zeta_2 \\ 
        &= 1 - x(-1) \\\
        &= x - \zeta_2\\
        &= \Phi_2(x)
    \end{align*}
    For $m > 2$, we note that $\phi(m)$ is equal to the number of $k$ from $1 \leq k \leq m$ where $\gcd(m, k) = 1$. Hence:

    $$x^{\phi(m)}\Phi_m(1/x) = \prod_{\substack{1 \leq k \leq m \\\gcd(m,k) = 1}}x \cdot \left(\frac{1}{x}-\zeta_m^k\right) = \prod_{\substack{1 \leq k \leq m \\\gcd(m,k) = 1}} (1-\zeta_m^kx) $$
    
    We note that if $m/2$ is an integer then $\gcd(m, m/2) \neq 1$. We then note that for every $k$ for $1 \leq k \leq m$ where $\gcd(k,m) = 1$, we get that $\gcd(m -k, m) = 1$ from Proposition 2.15. Since $m/2$ is not part of the $k$ that satisfies $\gcd(m, k) = 1$, we get that the $k$ where $\gcd(m, k) = 1$ can be paired and be split along $m/2$. Thus: 

    \begin{align*}
        \prod_{\substack{1 \leq k \leq m \\\gcd(m,k) = 1}} (1-\zeta_m^kx) &= \prod_{\substack{1 \leq k < m/2 \\\gcd(m,k) = 1}} (1-\zeta_m^kx) \prod_{\substack{m/2 < i \leq m \\\gcd(m,i) = 1}} (1-\zeta_m^ix) \\
        &= \prod_{\substack{1 \leq k < m/2 \\\gcd(m,k) = 1}} (1-\zeta_m^kx) (1 - \zeta_m^{m-k}x) 
    \end{align*}

    We then expand and refactor them to get: 

    \begin{align*}
        \prod_{\substack{1 \leq k < m/2 \\\gcd(m,k) = 1}} (1-\zeta_m^kx) (1 - \zeta_m^{m-k}x) &= \prod_{\substack{1 \leq k < m/2 \\\gcd(m,k) = 1}} (1 - \zeta_m^kx - \zeta_m^{m-k}x + x^2)\\
        &= \prod_{\substack{1 \leq k < m/2 \\\gcd(m,k) = 1}} (x-\zeta_m^k) (x-\zeta_m^{m-k})\\
        &= \prod_{\substack{1 \leq k < m/2 \\\gcd(m,k) = 1}} (x-\zeta_m^k) \prod_{\substack{m/2 < i \leq m \\\gcd(m,i) = 1}} (x-\zeta_m^i)\\
        &= \prod_{\substack{1 \leq k \leq m\\\gcd(m,k) = 1}} (x-\zeta_m^k) \\
        &= \Phi_m(x)
    \end{align*}

    as desired. \\

    \item 
    For $m \geq 2$, we prove by induction that $\Phi_{p^k}(1) = p$ for all $k \in \mathbb{N}$ for any prime $p$.

    For $k = 1$, we get that:

    $$\Phi_p(1) = \frac{1^p - 1}{1 - 1} = 1^{p-1} + 1^{p-2} + \cdots + 1 = p(1) = p$$

    We assume by strong induction that for $1 \leq i \leq m$ that $\Phi_{p^i}(1) = p$. For $k = m + 1$, we get that from Proposition 6.3 and that all of its divisors are $1, p,p^2,\ldots, p^m, p^{m+1}$ that: 

    \begin{align*}
    1^{p^{m+1}} -1 &= \Phi_1(1) \cdot \Phi_p(1) \cdots \Phi_{p^m}(1) \cdot \Phi_{p^{m+1}}(1) \\
    \frac{1^{p^{m+1}} -1}{1 - 1} &= \Phi_p(1) \cdot \Phi_{p^2}(1) \cdots \Phi_{p^m}(1) \cdot \Phi_{p^{m+1}}(1)\\
    \end{align*}

    By the induction hypothesis, we get that: 
    \begin{align*}
    1^{p^{m+1} -1} + 1^{p^{m+1} -2} + \cdots + 1^1 + 1 &= p^m \cdot \Phi_{p^{m+1}}(1) \\
    p^{m+1}(1) &= p^m \cdot \Phi_{p^{m+1}}(1) \\
    \Phi_{p^{m+1}}(1) &= p
    \end{align*} 
    Thus, for all $k \in \mathbb{N}$ and for any prime $p$, we get that $\Phi_{p^k}(1) = p$. \\

    We perform induction again for $m \geq 2$, but this time, we state the induction hypothesis as: 

    $$
    \Phi_m(1) =
    \begin{cases}
    p & \text{if } m = p^k \text{ for some prime } p \text{ and } k \in \mathbb{N}\\
    1  & \text{otherwise}\\
    \end{cases}
    $$

    The case where $m = p^k$ for some $p$ and $k \in \mathbb{N}$ is proven by our earlier induction. We concern ourselves with the otherwise case. The smallest otherwise case is when $m = 6$, and we show that:
    \begin{align*}
        \Phi_6(1) &= \frac{1^6-1}{\Phi_1(1)\cdot \Phi_2(1) \cdot \Phi_3(1)} \\
        &= \frac{1^5 + 1^4 + 1^3 + 1^2 + 1^1 + 1 }{2 \cdot 3  }\\
        &= \frac{6}{6}  = 1
    \end{align*}
    
    We now assume by strong induction that for $i$ where $6 \leq i \leq m$ that the induction hypothesis holds. For the case $m$, if $m$ can be expressed by a perfect prime power then it is proven. However, if $m$ is not, we first show that: 

    $$x^m - 1= \prod_{d \mid m} \Phi_d(x)$$

    Thus: 
    \begin{align*}
    \frac{1^m - 1}{1 - 1} &= \prod_{\substack{d \mid m \\ d \neq 1}} \Phi_d(1) \\
    1^{m-1} + 1^{m-2} + \cdots + 1^1 + 1 &= \prod_{\substack{d \mid m \\ d \neq 1}} \Phi_d(1) \\
    m &= \prod_{\substack{d \mid m \\ d \neq 1}} \Phi_d(1)
    \end{align*}

    We then note that if for some prime $p$ and $\nu_p(m) \geq 1$, we get that $p^1, p^2, \ldots,p^{\nu_p(m)}$ are all divisors of $m$. We also then note that $\Phi_{p^1}(1), \Phi_{p^2}(1), \ldots,\Phi_{p^{\nu_p(m)}}(1)$ are all equal to $p$ from our earlier induction, so multiplying them all gives us $p^{\nu_p(m)}$. This means isolating the divisors expressible as $p^k$ for some prime $p$ and some integer $k$ and multiplying their $\Phi_{p^k}(1)$ gives us the prime factorization of $m$, which equals $m$. Thus: 

    \begin{align*}
    m &= \prod_{\substack{p\\p\mid m}} p^{\nu_p(m)} \prod_{\substack{d \mid m \\ d \neq 1 \\ d \neq p^k }} \Phi_d(1) \\
    1 &= \prod_{\substack{d \mid m \\ d \neq 1 \\ d \neq p^k \\d \neq m }} \Phi_d(1) \cdot \Phi_m(1)
    \end{align*}


    The product outside of $\Phi_m(1)$ must all equal $1$ as all of its remaining divisors $d$ are $d < m$, which means their product is equal to $1$. Thus: 

    $$1 = 1 \cdot \Phi_m(1) = \Phi_m(1)$$

    This completes the induction, and we proved our desired result for $\Phi_m(1)$ for any $m \in \mathbb{N}$.
\end{enumerate}

\newpage

\item 
    \begin{enumerate}
    \item
    We assume it to be true. From Proposition 6.3 and since $m$ is a divisor of itself, we see that: 

    $$x^m - 1 = \Phi_m(x) \cdot \prod_{\substack{d \mid m \\ d \neq m}} \Phi_d(x)$$
    
    Thus, we get that $\Phi_m(69) = 420 \mid 69^m - 1$, so there exists an integer $q$ where $69^m -1 = 420q = 3(140)q$, so $3 \mid 69^m - 1$. This is a contradiction because $3 \mid 69$, so $3 \mid 69^m$. Thus, the statement is false. \\ 

    \item 
    We assume that there does exist such an integer $a$ where $23 \mid \Phi_{69}(a)$. \\
    
    If $a \equiv 1 \pmod{23}$, then we can apply LTE to $a^{69} - 1$ as $23$ is an odd prime and $23 \nmid 1$, $23 \mid a - 1$. Thus:
    \begin{align*}
    \nu_{23}(a^{69} - 1) &= \nu_{23}(a- 1) +\nu_{23}(69) \\
    &= \nu_{23}(a- 1) + 1
    \end{align*} 

    From Proposition 6.3, we see that 

    $$\Phi_{69}(a) = \frac{a^{69} - 1}{\Phi_3(a) \cdot  \Phi_{23}(a) \cdot \Phi_1(a)} $$

    We also note that since $23$ is a prime: 

    \begin{align*}
        \Phi_{23}(a) &= \frac{a^{23} - 1}{ a - 1} \\
        \Phi_{23}(a) \cdot \Phi_1(a) &= a^{23} - 1
    \end{align*}

    Thus, we then apply LTE on $a^{23} - 1$ to get that: 

    \begin{align*}
        \nu_{23}(\Phi_{69}(a)) &= 
        \nu_{23}(a^{69} - 1) - \nu_{23}(a^{23} - 1) - \nu_{23}(\Phi_3(a))\\
        &= \nu_{23}(a- 1) + 1 - (\nu_{23}(a- 1) + 1) - \nu_{23}(\Phi_3(a)) \\
        &= -\nu_{23}(\Phi_{3}(a))
    \end{align*}

    This implies that $\nu_{23}(\Phi_{69}(a)) = -\nu_{23}(\Phi_{3}(a)) \leq 0$. Thus, $23 \nmid \Phi_{69}(a)$, so $a \not\equiv 1 \pmod{23}$. \\

    If $a \equiv 0 \pmod{23}$, then $23 \nmid a^{69} - 1$ as $23 \mid a^{69}$. However, as mentioned in a), $\Phi_{69}(a) \mid a^{69} - 1$ from Proposition 6.3, which should imply $23 \mid a^{69} - 1$ if $23 \mid \Phi_{69}(a)$. This is a contradiction, so $a \not\equiv 0 \pmod{23}$. \\

    This leaves that $a \not\equiv 0, 1 \pmod{23}$. By Fermat's Little Theorem, we get that $a^{22} \equiv 1 \pmod{23}$. Since $23$ is a prime and $23 \nmid a$, they are coprime. Thus, from Proposition 5.7, we get that $o_{23}(a) \mid 22$. Thus, $o_{23}(a)$ must be either $1$, $2$, $11$, or $22$. However, $1$ is ruled out because that implies $a \equiv 1 \pmod{23}$. This means $23 \nmid a^{69} - 1$ because if it did then $a^{69} \equiv 1 \pmod{23}$, so $o_{23}(a) \mid 69$, which is not possible given its options. However, we assumed $23 \mid \Phi_{69}(a)$ and $\Phi_{69}(a) \mid a^{69} - 1$, so $23 \mid a^{69} - 1$, a contradiction. \\

    All possible congruences that $a$ can be to mod $23$ result in a contradiction. This means that the integer $a$ does not exist, so the statement is false. \\

    \item

    We note that for a given $m > 2$, we use our result from c) to show that:

    $$\Phi_m(x) = \prod_{\substack{1 \leq k \leq m/2 \\ \gcd(m, k) = 1}} (x - \zeta_m^k) (x - \zeta_m^{m-k})$$

    We expand them to get a similar result as in 1c), where 

    \begin{align*}
        \Phi_m(x) &= \prod_{\substack{1 \leq k \leq m/2 \\ \gcd(m, k) = 1}} x^2 - (\zeta^k_m + \zeta^{m-k}_m)x + 1 \\
        &= \prod_{\substack{1 \leq k \leq m/2 \\ \gcd(m, k) = 1}} x^2 - 2\cos(2k\pi/m)x + 1
    \end{align*}

    Thus, we also note that: 

    $$(x-1)^2 \leq  x^2 - 2\cos(2k\pi/m)x + 1\leq (x+1)^2$$

    Since our product was splitting $\phi(m)$ into ordered pairs of $(k, m -k)$, we note that there exist $\phi(m)/2$ terms in the product. Thus, we get that:

    $$(x-1)^{2\cdot\phi(m)/2}\leq \prod_{\substack{1 \leq k \leq m/2 \\ \gcd(m, k) = 1}} x^2 - 2\cos(2k\pi/m)x + 1 \leq (x+1)^{2 \cdot \phi(m)/2}$$
    $$(x-1)^{\phi(m)}\leq \Phi_m(x) \leq (x+1)^{\phi(m)}$$

    We now apply this result to $\Phi_{69}(420)$ and $\Phi_{420}(69)$. We first evaluate $\phi(69)$ and $\phi(420)$ using Exercise 5.1. 

    \begin{align*}
        \phi(69) &= \phi(23)\cdot\phi(3)\\
        &= (23 - 1)(3 - 1)\\
        &= 44 \\
        \phi(420) &= \phi(2^2)\phi(3)\phi(5)\phi(7)\\
        &= (4-2) (3-1) (5-1)(7-1)\\
        &= 96
    \end{align*}

    We then apply the result to get that: 

    $$(419)^{44} \leq \Phi_{69}(420) \leq (421)^{44}$$
    $$(68)^{96} \leq \Phi_{420}(69) \leq (70)^{96}$$

    We note that $96 \cdot \ln(68) > 44 \cdot\ln(421)$, so $68^{96} > 421^{44}$, thus $\Phi_{420}(69) > \Phi_{69}(420)$. This is opposite of what was stated. Thus, the statement is false. 
    
    \end{enumerate}

    \newpage

    \item
    \begin{enumerate}
        \item 
        Assume that this ring homomorphism $f: \mathbb{R} \rightarrow \mathbb{Q}$ exists. We note that $f(1) + f(1) = 2$, so $f(2) = 2$. Thus, we note that $f(\sqrt{2})^2=f(2)=2$ or $f(\sqrt{2})^2=2$, but there does not exist a number in $\mathbb{Q}$ that can satisfy the value for $f(\sqrt{2})$, which leads to a contradiction. \\

        \item 
        Assume this ring homomorphism $f: \mathbb{Q} \rightarrow \mathbb{Z}$ exists. Then, we see that $f(1) = 1$, so $f(1) + f(1) = f(2) = 2$. We then note that $f(1/2)\cdot f(2) = f(1) =1$, but that also implies $f(1/2) \cdot 2 = 1$. There does not exist a number in $\mathbb{Z}$ that satisfies the value for $f(1/2)$, which leads to a contradiction. \\

        \item 
        Since $a \geq b$, we note that $a - b \geq 0$. Thus, since $a-b$ is non-negative, we denote a non-negative number $x = \sqrt{a -b}$. We then note that $g(a) = g(b) + g(a - b) = g(b) + g(x)^2$. Since $g(x)^2$ must be non-negative, we get that $g(a) - g(b) \geq 0$, which implies $g(a) \geq g(b)$ as desired.\\

        \item 
        Let $r \in \mathbb{Q}$ be arbitrary. There exist integers $p$ and $q$ where $r = p/q$. Since $g$ is a ring homomorphism, we note that $g(1) - g(1) = 0$, so $g(0) = 0$, which gives $g(1) + g(-1) = 0$, thus $g(-1) = -1$. Thus, if $p < 0$, we add $g(-1)$ to itself $-p$ times. If $p > 0$, we add $g(1)$ to itself $p$ times. If $p = 0$, then $g(0) = p = 0$. In all cases, we get that $g(p) = p$, and doing the same with $q$ will yield $g(q) = q$. Since $g(1/q) \cdot g(q) = 1$, we get that $g(1/q) = 1/ g(q) = 1/q$. Hence, we get that $g(p) \cdot g(1/q) = p \cdot 1/q$ or $g(p/q) = g(r) = p/q = r$ as desired. 
    \end{enumerate}

    \newpage

    \item 
    \begin{enumerate}
        \item 
        Since addition and multiplication is defined coordinate-wise, we get the following given two arbitrary $(r_n)_{n=1}^\infty$ and  $(\hat{r}_n)_{n=1}^\infty$ in $R$: 
        \begin{align*}
            (r_n)_{n=1}^\infty + (\hat{r}_n)_{n=1}^\infty &= (r_n + \hat{r}_n)_{n=1}^\infty \\
            (r_n)_{n=1}^\infty \cdot (\hat{r}_n)_{n=1}^\infty &= (r_n \cdot \hat{r}_n)_{n=1}^\infty
        \end{align*}

        Thus, we get that $\pi_j((1)_{n=1}^\infty) = 1$ because the series is entirely composed of it regardless of $j$. Meanwhile, for addition, we get that: 

        \begin{align*}
            \pi_j((r_n)_{n=1}^\infty) + \pi_j((\hat{r}_n)_{n=1}^\infty) &= r_j + \hat{r}_j \\
            &= \pi_j((r_n + \hat{r}_n)_{n=1}^\infty )\\
            &= \pi_j((r_n)_{n=1}^\infty + (\hat{r}_n)_{n=1}^\infty)
        \end{align*}

        Meanwhile, for multiplication, we get that: 

        \begin{align*}
            \pi_j((r_n)_{n=1}^\infty) \cdot \pi_j((\hat{r}_n)_{n=1}^\infty) &= r_j \cdot \hat{r}_j \\
            &= \pi_j((r_n \cdot \hat{r}_n)_{n=1}^\infty )\\
            &= \pi_j((r_n)_{n=1}^\infty \cdot (\hat{r}_n)_{n=1}^\infty)
        \end{align*}

        Thus, $\pi_j$ satisfies all the conditions for a ring homomorphism, thus it is one. \\

        \item 
        Since $\varphi_{j+1}$ and $f_j$ are ring homomorphisms, we get that: 

        \begin{align*}
            \varphi_{j+1}(1_S) &= 1 \\
            f_j(1) &= 1 \\
            f_j(\varphi_{j+1}(1_S)) &= 1
        \end{align*}

        For addition and multiplication, we denote arbitrary $s, \hat{s} \in S$. For addition we get that: 

        \begin{align*}
            f_j(\varphi_{j+1}(\hat{s} + s)) &= f_j(\varphi_{j+1}(\hat{s}) + \varphi_{j+1}(s))\\
            &= f_j(\varphi_{j+1}(\hat{s})) + f_j(\varphi_{j+1}(s))
        \end{align*}

        Meanwhile, for multiplication, we get that: 

        \begin{align*}
            f_j(\varphi_{j+1}(\hat{s} \cdot s)) &= f_j(\varphi_{j+1}(\hat{s}) \cdot \varphi_{j+1}(s))\\
            &= f_j(\varphi_{j+1}(\hat{s})) \cdot f_j(\varphi_{j+1}(s))
        \end{align*}

        Thus, we showed that $f_j \circ \varphi_{j+1}$ satisfies all the properties for a ring homomorphism, thus it is one. \\

        \item 
        We define $\varphi:S \rightarrow R$ as: 

        $$\varphi(s) = (\varphi_j(s))^{\infty}_{j=1}$$

        We verify that $\varphi(s) \in R$ by showing that for any $j \in \mathbb{N}$, since $\varphi_j(s) = f_j\circ\varphi_{j+1}(s)$, we get that $f_j(\varphi_{j+1}(s)) = \varphi_j(s)$. We now show that it is also a ring homomorphism. \\
        
        To start, we first show that: 

        \begin{align*}
            \varphi(1_S) &= (\varphi_j(1_S))^{\infty}_{j=1} \\
            &= (1)^{\infty}_{j=1}
        \end{align*}

        For what follows, we denote arbitrary $s, \hat{s} \in S$. For addition, we get that: 
        \begin{align*}
            \varphi(s) + \varphi(\hat{s}) &= (\varphi_j(s))^{\infty}_{j=1} + (\varphi_j(\hat{s}))^{\infty}_{j=1}\\
            &= (\varphi_j(s) + \varphi_j(\hat{s}))^{\infty}_{j=1} \\
            &= (\varphi_j(s + \hat{s}))^{\infty}_{j=1}  \\
            &= \varphi(s + \hat{s})
        \end{align*}

        For multiplication, we get that: 
        \begin{align*}
            \varphi(s) \cdot \varphi(\hat{s}) &= (\varphi_j(s))^{\infty}_{j=1} \cdot (\varphi_j(\hat{s}))^{\infty}_{j=1}\\
            &= (\varphi_j(s) \cdot \varphi_j(\hat{s}))^{\infty}_{j=1} \\
            &= (\varphi_j(s \cdot \hat{s}))^{\infty}_{j=1}  \\
            &= \varphi(s \cdot\hat{s})
        \end{align*}

        Thus, we showed that $\varphi$ satisfies all the conditions for a ring homomorphism, thus it is one. We note that $\pi_j((\varphi_j(s))^{\infty}_{n=1}) = \varphi_j(s)$. Hence, we proved the existence of a $\varphi:S \rightarrow R$ where $\pi_j \circ \varphi = \varphi_j$. 
        
    \end{enumerate}

    
    
\end{enumerate}
\end{document}