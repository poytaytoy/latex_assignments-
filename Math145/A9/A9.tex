    \documentclass{article}
    \usepackage{enumitem}
    \usepackage{mathtools,amssymb,amsthm} % imports amsmath
    \usepackage{enumitem}
    \usepackage[a4paper, total={6in, 9in}]{geometry}
    \newtheorem{remark}{Remark}

    \begin{document}

    \begin{enumerate}

    \item 
    \begin{enumerate}
        \item 
        If $p = 2$, we note that $3 \equiv 1 \pmod{2}$, so $3$ is a quadratic residue mod $2$ since $1$ is a square. If $p = 3$, then $3 \equiv 0 \pmod{3}$ since $0$ is a square. \\

        Hence, it remains to prove the claim if $p$ is an odd prime where $p \geq 5$. Let us assume $3$ is a quadratic residue mod $p$. Hence, $(\frac{3}{p})$ = 1. We consider 2 cases on the bases of Theorem 11.4 (c).\\

        \textbf{Case 1: }If $p \equiv -1 \pmod{4}$. Then, $(\frac{3}{p}) = -(\frac{p}{3})$ = 1. Thus, $(\frac{p}{3}) = -1$. This implies that $p^{(3-1)/2} = p \equiv -1 \pmod{3}$. Since $3$ and $4$ are co-prime, we get that $p \equiv -1 \pmod{12}$. \\

        \textbf{Case 2: }If $p \not \equiv -1 \pmod{4}$. Then, $(\frac{3}{p}) = (\frac{p}{3})$ = 1. This implies that $p^{(3-1)/2} = p \equiv 1 \pmod{3}$. Since $p$ is an odd prime, we note that $3 \not \equiv 2 \pmod{4}$ or $3 \not \equiv 0 \pmod{4}$. This leaves that $3 \equiv 1 \pmod{4}$. Since $3$ and $4$ are co-prime, we get that $p \equiv 1 \pmod{12}$. \\

        Considering all the cases, we get that $(\frac{3}{p}) = 1$ if $p = 2, 3$ or $p \equiv \pm 1 \pmod{12}$ Proving the converse is trivial as reversing the proof will show that if $p = 2, 3$ or $p \equiv \pm 1 \pmod{12}$, $p$ would be a quadratic residue mod $3$. \hfill $\square$  \\

        \item 
        For $p = 2$, we note that $-3 \equiv -1 \equiv 1 \pmod{2}$, so it is a quadratic residue. Meanwhile for $p = 3$, $-3 \equiv 0 \pmod{3}$, so it is also a quadratic residue. Hence, we concern ourselves with the case where $p \leq 5$. This implies that $(\frac{-3}{p}) = (\frac{-1}{p})(\frac{3}{p}) = 1$. In other words, either both $(\frac{-1}{p})$ and $(\frac{3}{p})$ must be equal. Hence: 

        $$(\frac{-1}{p}) = (\frac{3}{p})$$
        
        Applying Theorem 11.4 (a), we get that: 


        $$(-1)^{(p-1)/2} = (\frac{3}{p})$$

        We now consider Theorem 11.4 (c), and note that

        $$(\frac{3}{p}) \cdot (\frac{p}{3}) = (-1)^{(p-1)/2 \cdot (3 - 1)/2} = (-1)^{(p-1)/2}$$ 

        We know that $(\frac{3}{p}) \neq 0$ because $p$ is prime and $p \neq 3$, so: 
        
        $$(\frac{3}{p}) = (\frac{p}{3}) \cdot (-1)^{(p-1)/2}$$

        Substituting the value with the earlier equation gives: 
        \begin{align*}
            (-1)^{(p-1)/2} &= (\frac{p}{3}) \cdot (-1)^{(p-1)/2} \\
            1 &= (\frac{p}{3}) 
        \end{align*}

        This implies that $p$ is a square mod $3$. In $\mathbb{F}_3$, only $0$ and $1$ are squares. We can rule out $p = 0$ because $p \neq 3$. Hence, $-3$ is a quadratic residue mod $p$ if $p = 2,3$ or $p \equiv 1 \pmod{3}$. Proving the converse is again trivial as reversing the proof should suffice to show that if $p = 2,3$ or $p \equiv 1 \pmod{3}$, $-3$ is a quadratic residue mod $p$. \hfill $\square$  \\

        \item 
        By contradiction, we assume that $p \neq 2$ and $p \not \equiv 1 \pmod{4}$. This implies that $p$ is an odd prime where $2 \nmid (p-1)/2$. We note that $x^2 \equiv -y^2 \pmod{p}$. This implies that $-y^2$ is a quadratic residue mod $p$, so either $(\frac{-y^2}{p}) = 0$ or $1$. If $(\frac{-y^2}{p}) = 1 = (\frac{-1}{p})(\frac{y}{p})(\frac{y}{p})$, we note that $(\frac{-1}{p}) = -1$ since $(p-1)/2$ is odd. This implies that $(\frac{y}{p})^2 = -1$. This equation does not have a real solution, a contradiction. Meanwhile, if $(\frac{-y^2}{p}) = 0$, then with $(\frac{-1}{p}) = -1$, $(\frac{y}{p}) = 0$. This implies $p \mid y$, but if $y \neq 0$, then $x^2 + y^2 \geq y^2 > p$, a contradiction, and if $y = 0$, then $p = x^2$, contradicting $p$ being prime. In all cases, we reach a contradiction thus $p = 2$ or $p \equiv 1 \pmod{4}$. \hfill $\square$ \\

        \item 
        By contradiction, we assume that $p \neq 3$ and $p \not \equiv 1 \pmod{3}$. This leaves that $p \equiv -1 \pmod{3}$. Now we consider that: 
        \begin{align*}
            x^2 -xy + y^2 &= p \\
            4x^2 - 4xy + 4y^2 &= 4p \\
            (2x - y)^2 + 3y^2 &= 4p 
        \end{align*}

        We now consider that: 

        $$(2x - y)^2 + 3y^2 \equiv 4p \pmod{3}$$

        In $\mathbb{F}_3$, $1 = 4$ and $3 = 0$. Hence: 

        $$(2x - y)^2 \equiv p \pmod{3}$$

        However, this implies that $(2x - y)^2 \equiv -1 \pmod{3}$. This is a contradiction because $-1$ is not a square in $\mathbb{F}_3$ because $(\frac{-1}{3}) = (-1)^{(3-1)/2} = -1$. \hfill $\square$ 

    \end{enumerate}


    \newpage 

    \item 
    \begin{enumerate}
        \item
        By contradiction, assume $n$ is even then $n = 2k$ for some integer $k$. We then note that $2^n - 1 = 4^k - 1$. Note that $4 \equiv 1 \pmod{3}$, so $4^k \equiv 1 \pmod{3}$. Thus, $3 \mid 4^k - 1$ thus $3 \mid 3^n -1$. However, we note that $3 \nmid 3^n - 1$, contradiction. \hfill $\square$  \\

        \item 
        We note that $3^n \equiv 1 \pmod{p}$. This implies that $3^n$ is a quadratic residue mod $p$, so $(\frac{3^n}{p}) = 1$. We then note that we can split $(\frac{3^n}{p})$ into $n$ products of $(\frac{3}{p})$. However, $n$ is odd, so $(\frac{3}{p}) \neq 0, -1$. This leaves that $(\frac{3}{p}) = 1$. By 1(a), this implies that $p = 2, 3$ or $p \equiv \pm 1 \pmod{12}$. However, $p \neq 2$ because $2$ is even. Meanwhile, $p \neq 3$ because $3 \nmid 3^n -1$. This leaves that $p \equiv \pm 1 \pmod{12}$ as desired. \hfill $\square$  \\
        
        \item 
        For $n = 1$, we note that $1 \mid 2$, so it works. We now assume that $n > 1$. Since from $(a)$, we get that $n$ is odd, so we can express $n = 2k + 1$ for some integer $k$. Hence, $2^n - 1 = 2 \cdot 4^k - 1$. We then note that $4^2 \equiv 4 \pmod {12}$. This implies that for $k \geq 1$, we get that $4^k \equiv 4 \pmod {12}$ and that $2 \cdot 4^k - 1 \equiv 2\cdot 4 - 1\equiv 7 \pmod {12}$. However, we note that since $n > 1$ and is odd, $2^n - 1 \geq 7$ and is also odd. This implies that $2^n - 1$ can be prime factorized into entirely of odd primes of $p_1p_2 \cdots p_j$. Note that any odd prime $p_i$ has $p_i \mid 2^n -1$, so it is also $p_i \mid 3^n - 1$. From $(b)$, this implies that $p_i \equiv \pm 1 \pmod {12}$. This is a contradiction because $p_1p_2 \cdots p_j \equiv \pm 1 \pmod {12}$ and that $7 \not \equiv \pm 1 \pmod {12}$. Thus, it must be that $n = 1$. \hfill $\square$ 
        
    \end{enumerate}

    \newpage 

    \item 
    \begin{enumerate}
        \item 
        We first note that if $p \equiv 1 \pmod 3$, from 1(b), we note that $-3 \equiv a^2 \pmod p$ for some integer $a$. We then note that if $a$ is even, we note that $p-a$ is odd and that $p - a \equiv a \pmod p$. Hence this conversion allows us to assume there must exist an odd $a$. Since $a$ is odd, it is in the form $a = 2k + 1$ for some integer $k$. This gives: 
        \begin{align*}
            (2k+1)^2 &\equiv -3 \pmod p \\ 
            4k^2 + 4k + 1 + 3 &\equiv 0 \pmod p \\
            4(k^2 + k + 1) &\equiv 0 \pmod p
        \end{align*}

        Since $p \equiv 1 \pmod 3$, we may assume $p \neq 2$ thus an odd prime. Hence, $p \nmid 4$ and so $4 \neq 0 \in \mathbb{F}_p$. From the above expression, we get that $4(k^2 + k + 1) = 0$ in $\mathbb{F}_p$. Since $\mathbb{F}_p$ is an integral domain, it must be that $k^2 + k + 1 = 0$ in $\mathbb{F}_p$. We note that the formal derivative of this polynomial is $2x + 1$. We aim to prove that $2k + 1 \neq 0$ in $\mathbb{F}_p$. By contradiction, $2k + 1 = 0$ in $\mathbb{F}_p$. We get that $k = -2^{-1}$ $\mathbb{F}_p$ (Note that $p \neq 2$, so $2^{-1}$ exists). Hence:
        \begin{align*}
            4k^2 + 4k + 4 &= 0 \\
            4(-2^{-1})^2 + 4 (-2^{-1}) + 4 &= 0 \\
            1 - 2 + 4 &= 0  \\
            3 &= 0 
        \end{align*}

        Since $p \equiv 1 \pmod 3$, we note that $p \neq 3$. This is a contradiction, so it must be that $2k + 1 \neq 0$ in $\mathbb{F}_p$. We satisfied the conditions for Corollary 11.16, so there exists a $a \in \mathbb{Z}_p$ where $a^2 + a + 1 = 0$ in $\mathbb{Z}_p$. In other words, by taking the $k$-th coordinate of $a$ as $a_k$, we note that $a_k^2 + a + 1 = 0$ in $\mathbb{Z}/p^k\mathbb{Z}$. Hence, there is a solution for $x^2 + x + 1 \equiv 0 \pmod {p^k}$. \hfill $\square$  \\

        \item 
        Let us assume $p$ is a prime where $p \equiv 2 \pmod 3$. \\
        
        \textbf{Claim: }If $a$ is an integer where $p \nmid a$, then there exist an integer solution for $x^2 \equiv a \pmod p$. \\

        We first note that $p = 3k + 2$ for some integer $k$. By Fermat's Little Theorem, we note that $a^{p-1} \equiv 1 \pmod p$ and $a^{p} \equiv a \pmod p$. Multiplying the two gives that $a^{2p - 1} \equiv a \pmod p$. We then substitute $p$ to get that $a^{6k + 3} \equiv (a^{(2k + 1)})^3  \equiv a \pmod p$. We note that $a^{(2k + 1)}$ is the solution, so we conclude the proof. \hfill $\square$ \\

        We now note that $19 \equiv 1 \pmod 3$, so since $19$ is a prime, we get that $p \nmid 19$. Hence, by the claim, there is an integer solution $x$ to $x^3 \equiv 19 \pmod p$. In other words, there is a solution to $x^3 - 19 \equiv 0 \pmod p$, or that $x^3 - 19 = 0$ in $\mathbb{F}_p$. The formal derivative for $x^3 - 19$ is $3x^2$. We aim to prove that $3x^2 \neq 0$ in $\mathbb{F}_p$. By contradiction, $3x^2 = 0$ in $\mathbb{F}_p$. We first note that $\mathbb{F}_p$ is an integral domain and that $3 \neq p$ thus $3 \neq 0$ in  $\mathbb{F}_p$, so $x^2 = 0$ and consequently, $x = 0$ in $\mathbb{F}_p$. However, this implies that $x^3 \equiv 0^3 \equiv 19 \pmod p$, a contradiction as $p \nmid 19$. Thus, it must be that $3x^2 \neq 0$ in $\mathbb{F}_p$. \\
        
        By Corollary 11.16, there exists a $a \in \mathbb{Z}_p$ where $a^3 - 19 = 0$ in $\mathbb{Z}_p$. In other words, by taking the $k$-th coordinate of $a$ as $a_k$, we note that $a_k^3 - 19 = 0$ in $\mathbb{Z}/p^k\mathbb{Z}$. Hence, there is a solution for $x^3 - 19 \equiv 0 \pmod {p^k}$. \hfill $\square$  \\

        \item 
        Let us use $7 \in \mathbb{Z}_p$ and denote $f(x) = x^3 - 19$. We note that $f(7) = 7^3 - 19 = 324$ and that $f'(7) = 3 \cdot (7)^2 = 147$ in $\mathbb{Z}_p$. Note that using the definition from HW6 6 (c):
        
        $$324 = 0 + 0 + 0 + 0 + 1 \cdot 3^4 + 1 \cdot 3^5 + 0 + \cdots$$ 
        
        $$147 = 0 + 1 \cdot 3^1 + 1 \cdot 3^2 + 2 \cdot 3^3 + 1 \cdot 3^4 + 0 + \cdots$$

        This implies that $\nu_3(324) = 4$ and $\nu_3(147) = 1$. In other words, $\nu_3(f(7)) > 2\nu_3(f'(7))$. We can apply Hensel's Lemma and get that there exist a $\alpha \in \mathbb{Z}_p$ where $f(\alpha) = 0$. Take the $k$-th coordinate of $\alpha$ as $a_k$ and we see that $a_k^3 - 19 = 0$ in $\mathbb{Z}/3^k \mathbb{Z}$. This translates to that there exist an integer solution to $x^3 - 19 \equiv 0 \pmod {3^k}$. \hfill $\square$  

    \end{enumerate}

    \newpage 

    \item 
    \begin{enumerate}
        \item 
        We denote $f(x) = x^{46} + 69x + 23$. We first note that both $23 \mid 69$ and $23 \mid 23$. Thus, we note that in $\mathbb{F}_{23}[x]$, $x^{46}+69x+23 = x^{46}$. By contradiction, let us assume that the polynomial is reducible and that $f(x) = g(x)h(x)$ in $\mathbb{Z}[x]$. This implies that $g(x)h(x) = x^{46}$ in $\mathbb{F}_{23}[x]$. \\ 

        \textbf{Claim: } At least $g(x)$ or $h(x)$ is equal to perfect powers, meaning polynomials in the form $x^k$ in $\mathbb{F}_{23}[x]$. \\
        
        By contradiction, they are both equal to the form $h(x) = x^m a_{m-1}x^{m-1} + \cdots + a_rx^r$ and $g(x) = x^n + b_{n-1}x^{n-1} + \cdots + b_sx^s$ where all $a_r$ and $b_r$ are non-zeros in $\mathbb{F}_{23}$. We then get $g(x)h(x) = x^{mn} + \cdots b_sa_rx^{r + s}$. Since $\mathbb{F}_{23}$ is an integral domain, we get that $b_sa_rx^{r + s}$ is a non-zero term, which implies $f(x)$ cannot be equal to a perfect power, a contradiction. If only one of them is equal to a perfect power, where we assume $h(x) = x^m a_{m-1}x^{m-1} + \cdots + a_rx^r$ and $g(x) = x^n$ where $a_r$ is non-zero in $\mathbb{F}_{23}$. This means $h(x)g(x) = x^{mn} + \cdots a_rx^{r+n}$ and the same contradiction follows with the non-zero term $a_rx^{r+n}$. Hence, both $g(x)$ and $h(x)$ are equivalent to perfect powers in $\mathbb{F}_{23}[x]$. \hfill $\square$  \\

        Since $g(x) = x^m$ and $h(x) = x^n$ in $\mathbb{F}_{23}$, we note that $g(0) = 0$ and $h(0) = 0$ in $\mathbb{F}_{23}$. This implies that in $\mathbb{Z}$, $23 \mid g(0)$ and $23 \mid h(0)$, which implies $23^2 \mid f(0)$. However, we note that $f(0) = 23$ and $23^2 \nmid 23$, a contradiction. Thus, it must be that $f(x)$ is irreducible in $\mathbb{Z}[x]$.  \hfill $\square$  \\ 



        \item 
        We consider $f(x) = x^{46} + 69x + 2025$. Hence, we note that $f(0) = 2025$ and $f'(0) = 46\cdot 0^{45} + 69  = 69$ in $\mathbb{Z}_3$. Note that using the definition from HW6 6 (c): 

        $$2025 = 0 + 0 + 0 + 0 + 1 \cdot 3^4 + 2 \cdot 3^5 + 2 \cdot 3^6 + 0 + \cdots$$
        $$69 = 0 + 2 \cdot 3^1 + 1 \cdot 3^2 + 2 \cdot 3^3 + 0 + \cdots$$

        This implies that $\nu_3(2025) = 4$ and $\nu_3(69) = 1$. In other words, $\nu_3(f(0)) > 2\nu_3(f'(0))$. We apply Hensel's Lemma and get that there exist a $\alpha \in \mathbb{Z}_3$ where $f(\alpha) = 0$. In other words, we found such root. \hfill $\square$  


    \end{enumerate}
    \end{enumerate}

    \end{document}