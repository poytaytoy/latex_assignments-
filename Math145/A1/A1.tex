\documentclass{article}
\usepackage{mathtools,amssymb,amsthm} % imports amsmath
\usepackage{enumitem}
\usepackage[a4paper, total={6in, 9in}]{geometry}
\begin{document}


\begin{enumerate}[leftmargin=*, label=\arabic*.]

  \item
  \begin{enumerate}[label=\alph*)]
        \item 
        Let $x, y$ be integers where $d = ax + by$. If $e|b$ and $e|a$, then $e|ax$ and $e|by$, so $e|ax + by$ thus $e|d$ as desired. \\ 
        
        \item 
        We denote the set $S = \{ax + by : x, y \in \mathbb{Z}, ax + by > 0\}$. If $a < 0$ or $b < 0$, set $x_0$ or $y_0$ as -1 or else 1. The resulting sum of $ax_0 + by_0$ is positive thus in $S$. By the Well-Ordering Principle, since S is non-negative and non-empty, let $d = min(S)$ exists. By our construction of S, d is also the smallest positive integer in the form $ax + by$. \\
        
        By contradiction, $d \nmid a$, then by the Division Algorithm, there exist integers $q,r$ where $0 \leq r < |d|$ that $r = a  - dq$ and $r = a - q(ax + by) = a(1-qx)+b(-qy)$. Since $d \nmid a$, $r \neq 0$ and $0 < d$, so $0 < r < d$. This implies $r \in S$ and contradicts the minimality of d. This means $d|a$. Conversely, the same contradiction applies for $d\nmid b$, so $d|b$. Hence, $d|a$ and $d|b$ as desired. 
      
  \newpage
  
  \item
  \begin{enumerate}[label=\alph*)]
        \item 
        By following the algorithm, we see that $(69, 2025) \xrightarrow{} (24, 69) \xrightarrow{} (21, 24) \xrightarrow{} (3, 21) \xrightarrow{} (0, 3)$. Thus, $\ell(69, 2025) = 5$. \\
        
        \item
        Consider for $\ell(2025^{69} - 1, 2025^{420} - 1)$ that: \\ 

        $(2025^{69} - 1)(2025^{351} + 2025^{282} + 2025^{213} + 2025^{144} + 2025^{75} + 2025^{6})$ \\
        $= 2025^{420} - 2025^{6}$\\
        $= (2025^{420} - 1) - (2025^{6} - 1)$\\

        Thus: 
        $$2025^{420} - 1 = (2025^{69} - 1)(2025^{351} + 2025^{282} + ... + 2025^6) + (2025^{6} - 1)$$

        And continuing the process with $(2025^{69} - 1)$: 

        $$2025^{69} - 1 = (2025^{6} - 1)(2025^{63} + 2025^{57} + ... + 2025^3) + (2025^{3} - 1)$$

        And finally, with $(2025^{3} - 1)$:
        
        $$(2025^{3} - 1)(2025^{3}+1) = (2025^{6} - 1) - (0)$$
         
        With these information, we can see that $(2025^{69} - 1, 2025^{420} - 1) \xrightarrow{} (2025^{6} - 1, 2025^{69} - 1) \xrightarrow{} (2025^{3} - 1, 2025^{6} - 1) \xrightarrow{} (0, 2025^{3} - 1)$. Hence, $\ell(2025^{69} - 1, 2025^{420} - 1) = 4$. \\

        \item
        We prove that for all $a$ and for all $b \geq a , \ell(a, b) \leq 2\log_2(a)+2$. For the base case $a = 1$, let b be an arbitrary positive integer where $b \geq 1$, we see that $(1, b) \xrightarrow[]{} (0, 1)$. For all $b \geq 1, \ell(1, b) = 2$, and $2 = 2\log_2(1) + 2$.\\

        For the inductive hypothesis, we assume every integer $m$ where $1 \leq m < k$, for all $b \geq m,  \ell(m, b) \leq  2\log_2(m) + 2$. We consider the case k. Let $b$ be an arbitrary positive integer where $b \geq k$ and $r$ be, from the Division Algorithm, where $r = b - kq$ for some integer q and $0 \leq r < b$. We consider all cases for r. \\

        If r = 0, then $(k, b) \xrightarrow{} (0, k)$, so $\ell(k, b) = 2$ and $2 < 2\log_2(k) + 2$ since $ k > 1$. \\
        
        If $1 \leq r \leq \frac{k}{2}$, then $(k, b) \xrightarrow{} (r, k)$ or $1 + \ell(r, k)$. By the induction hypothesis, \\$\ell(r, k) \leq 2\log_2(r) + 2$ and $r \leq \frac{k}{2}$, thus: 

            \begin{align*}
              1 + \ell(r, k) &\leq 2\log_2(r) + 2 + 1 \\  
              1 + \ell(r, k) &\leq 2(\log_2(r) + 0.5) + 2 \\
              \ell(k, b) \leq 2\log_2(\sqrt{2}r) + 2 &\leq 2 \log_2(\frac{\sqrt{2}}{2}k) + 2 \\
              \ell(k, b) &< 2\log_2(k) + 2 \\
        \end{align*}
      
        If $r > \frac{k}{2}$, $\lfloor \frac{k}{r} \rfloor = 1$, so the remainder of r divided by k is equal to $k - r$ as $(k - r) = k - r(1)$. This means $(k, b) \xrightarrow{} (r, k) \xrightarrow{} (k - r, r)$. Thus, $\ell(k, b) = 2 + \ell(k -r, r)$. By the induction hypothesis, $\ell(k - r, r) \leq 2\log_2(k - r) + 2$ and since $k - r < \frac{k}{2}$, we get: 
        
        \begin{align*}
              2 + \ell(k - r, r) &\leq 2\log_2(k - r) + 2 + 2 \\
              2 + \ell(k - r, r) &\leq 2(\log_2(k - r) + 1) + 2 \\
              \ell(k, b) &\leq 2\log_2(2(k - r)) + 2 \\ 
              \ell(k, b) &< 2\log_2(k) + 2 
        \end{align*}
        
        Hence, for all $k \geq 2$, $\ell(k, b) \leq 2\log_2(k) + 2$ holds true. By induction on a, the bound $\ell(a, b) \leq 2\log_2(a) + 2$ holds for all $1 \leq a \leq b$. \\
        
    \end{enumerate}
    \newpage

  \item 

  By Proposition 2.13, $gcd(a, c)|c$ and $gcd(a, c)|a$, so the fractions $\frac{-c}{ad - bc}$ and $\frac{a}{ad - bc}$ are indeed integers. Thus: 

  $$(an + b)(\frac{-c}{ad - bc}) + (cn + d)(\frac{a}{ad-bc}) = \frac{-can - bc + can + ad}{ad-bc} = \frac{ad-bc}{ad-bc} = 1$$

  Hence, there exist integers $x$ and $y$ where $(an + b)x + (cn+d)y = 1$. By Corollary 2.18, the existence of such $x$ and $y$ implies that the $gcd(an+b, cn+d) = 1$ as desired. \\

  \newpage

  \item 

  \begin{enumerate}[label=\alph*)]
        \item 
        The remainder is going to be 1. \\
        
        \item
        To start, we convert the hints in the form of congruences: 

        $$
        (625 \equiv 2 \mod 89) \quad
        (800 \equiv -1 \mod 89) \quad
        (2^{11} \equiv 1 \mod 89)
        $$
        
        We note that $625 \cdot 18 - 800 \cdot 9 = 2(2025)$. Under Lemma 3.9, we add $(625 \cdot 18 \equiv 36 \mod 89)$ and $(800 \cdot -9 \equiv 9 \mod 89)$, so $(4050 \equiv 45 \mod 89)$.  By Exercise 3.5, let m,a,b,c be integers and we see that $(45 \cdot 90 \equiv 45 \cdot 1 \mod 89)$ can be expressed in the form $(ac \equiv bc \mod m)$. Since $gcm(45, 89) = 1$ and $\frac{89}{gcm(45, 89)} = 1$, so we get:
        
        $$90 \equiv 1 \mod \frac{89}{gcm(45, 89)}$$
        $$2 \cdot 3^2 \cdot 5 \equiv 1 \mod 89$$

        Using Lemma 3.9 again and multiplying by itself 22 times, we get $(2^{22} \cdot 3^{44} \cdot 5^{22} \equiv 1 \mod 89)$. We do the same with $(2^{11} \equiv 1 \mod 89)$ and multiply by itself to get $(2^{22} \equiv 1 \mod 89)$. Using congruence's symmetric property for $(1 \equiv 2^{22} \mod 89)$ and applying its transitive property to $(2^{22} \cdot 3^{44} \cdot 5^{22}\equiv 1 \mod 89)$ and $(1 \equiv 2^{22} \mod 89)$, we get:
        
        $$2^{22} \cdot 3^{44} \cdot 5^{22} \equiv 2^{22} \mod 89$$

        This implies $89 | 2^{22} \cdot (3^{44} \cdot 5^{22} - 1)$. Since $89 \nmid 2^{22}$ and 89 is a prime, by Euclid's Lemma, $89 | 3^{44} \cdot 5^{22} - 1$. Since $3^{44} \cdot 5^{22} = 2025^{11}$, we get $89| 2025^{11} - 1$ as desired. \\
        
        \item 
        To start, we note that $11 \cdot 184 + 1 = 2025$. Thus, for $2025^{11} - 1:$\\

        $= 2025^{10}(11 \cdot 184 + 1) - 1$ \\
        $= 2025^{10}\cdot 11 \cdot 184 + 2025^{10} - 1 $\\
        $= 2025^{10}\cdot 11 \cdot 184 + 2025^{9}(11\cdot 184 + 1) - 1 $\\
        $= 2025^{10}\cdot11\cdot184 + 2025^{9}\cdot11\cdot184 + ... + 2025^2\cdot11\cdot184 + 2025\cdot11\cdot184 + 11\cdot184 + 1 - 1$
        $= 11 \cdot 184 (2025^{10} + 2025^9 + ... + 2025^1 + 1 )$ \\
        
        Since $(2025 \equiv 1 \mod 11)$, by Lemma 3.9, each element in the sum expressible by $2025^k$ for some $k \in \mathbb{N}$ when multiplying $(2025 \equiv 1 \mod 11)$ by itself k times results in $(2025^k \equiv 1 \mod 11)$. Meanwhile, $1 \equiv 1 \mod 11$, so applying Lemma 3.9 again and summing all the 11 terms in the sum, we get: 

        $$2025^{10} + 2025^9 + ... + 2025^1 + 1 \equiv 11 \mod 11$$

        Since $11|11$ and $11|(2025^{10} + 2025^9 + ... + 2025^1 + 1) - 11$, it implies that: 
        
        $$11|(2025^{10} + 2025^9 + ... + 2025^1 + 1)$$

        Thus, there exist $q \in \mathbb{Z}$ where $11q$ is equal to the sum. Hence, $2025^{11} - 1 = 11\cdot184(11q)$, which gives us the $11^2|2025^{11} - 1$ as desired.
    \end{enumerate}
    
\end{enumerate}

\end{document}
