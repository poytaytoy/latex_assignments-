\documentclass{article}
\usepackage{enumitem}
\usepackage{mathtools,amssymb,amsthm} % imports amsmath
\usepackage{enumitem}
\usepackage[a4paper, total={6in, 9in}]{geometry}
\newtheorem{remark}{Remark}

\begin{document}
\begin{enumerate}
\item 
    \begin{enumerate}
        \item 
        Let $a$ be an arbitrary element in $H$ since it is non-empty. We note that $a^{|G|} = e$ by Corollary 12.4, so $e \in H$. This implies that $a^{|G| - 1} = a^{-1}$, so $a^{-1} \in H$. Lastly, for any $a, b \in H$, we get that $ab \in H$. Thus, we proved that $H$ is a subgroup of $G$. \hfill $\square$  \\

        \item 
        \textbf{Claim 1: } Let $a \in G$. If $o(a) = d$, then if there exist a $k \in \mathbb{N}$ where $a^k = e$ then $d \mid k$. \\

        By contradiction, $d \nmid k$, then by the Division Algorithm, there exist $q, r \in \mathbb{Z}$ where $a^k = a^{qd} \cdot a^r = e$ where $0 < r < d$. However, $a^{qd} = e^q = e$, so $a^r = e$. This contradicts the minimality of $o(a)$. Hence, $r$ must be 0 and $d \mid k$. \\

        \textbf{Claim 2: } Let $a \in G$. For any $k \in \mathbb{N}$, we have $o(a^k ) = \frac{o(a)}{\gcd(o(a), k)}$. \hfill $\square$  \\

        Let $d = o(a)$ and $w = d/\gcd(d, k)$. Then $wk = d(k/ \gcd(d,k))$ is divisible by $d$ and so $a^{kw} = e$. Furthermore, suppose we have a $n \in \mathbb{N}$ where $a^{kn} = e$. By Claim 1, we get that $d \mid kn$, so:

        $$\frac{d}{\gcd(k, d)} \mid \frac{k}{\gcd(k, d)}n$$

        We note that $\gcd(d / \gcd(k, d), k / \gcd(l, d)) = 1$. By Corollary 2.20, $w \mid n$. This implies the smallest value for $n$ is $w$. (Yes, this is just a re-write of the solution for HW3 2(b)) \hfill $\square$  \\

        \textbf{Claim 3: } For any positive integer $d \mid m$, there are exactly $\phi(d)$ elements of $G$ with order $d$ \\

        Since $G$ is cyclic, there exist some $g \in G$ with $o(g) = m$ where $G = \{g, g^2, \cdots, g^m \}$. For any $d \mid m$, there exist an integer $q$ where $dq = m$. Let $a = g^q \in G$ thus $a^d = g^{dq} = e$ and $o(a) = d$. Otherwise, $o(a) < d$ implies $q \cdot o(a) < qd = m$, contradicting the minimality of $m$. Since $o(a) = d$, we note that $a, a^2, \cdots, a^d$ are all unique because if there exist $a^j = a^k$ for $1 \leq j < k \leq d$, then $e = a^{k - j}$, contradicting $d$'s minimality. We apply Claim 2 to get that for $1 \leq k \leq d$ that $o(a^k) = d / \gcd(d, k)$. It is clear that if $o(a^k) = q$, then $\gcd(d, k) = 1$. We showed all of the $a^k$ are unique, so there exist at least $\phi(d)$ elements with order $d$. \\

        We now prove all orders $b \in G$ with order $d$ must be in the form $a^k$. Let $b = g^r$ for $1 \leq r \leq m$. We apply Claim 2 again to get that $o(g^r) = m / \gcd(m, r)$ thus $\gcd(m, r) = d / m = q$. This further implies that $q \mid r$, so there exist an integer $k$ where $1 \leq k \leq d$ that $qk = r$. Thus, $g^{r} = (g^q)^k = a^k$. This proves that there exist exactly $\phi(d)$ elements with order $d$. \hfill $\square$ \\

        \textbf{Claim 4: }For any positive integer $d \mid m$, there is a unique subgroup of G of order $d$. \\

        Let $a \in G$ where $o(a) = d$ and $a = g^{m/d}$, which exits by Claim 3. We then denote the set $H = \{a, a^2, \cdots a^d\}$. For any $1 \leq i, j \leq d$, $a^i \cdot a^j = a^{i + j}$. If $i + j \leq d$, then $a^{i + j} \in H$. Otherwise, we note that $a^{i + j} = a^{i + j - d} \cdot a^d = a^{i + j - d}$. Since $i + j \leq 2j$ so $i + j - d \leq d$, $a^{i + j} \in H$. By $(a)$, $H$ is a subgroup of $G$. Note that all elements in $H$ are unique by our proof of Claim 3, so it is also order $d$. \\

        It remains to prove $H$ is the only subgroup of order $d$ in $G$. Suppose there exist a subgroup $E \leq G$ with order $d$. Let $b \in E$, then $b^{d} = e$ by Corollary 12.4. We note that $b = g^r$ for some $r \in \mathbb{N}$ and $g^{rd} = e$. By Claim 1, $m \mid rd$, which implies that $m/d \mid r$. Hence, there exist some integer $k$ where $1 \leq k \leq d$ and $b = g^r = (g^{m/d})^k = a^k$. Hence, $b \in H$ and $E \subseteq H$. Since they have the same order, by the pigeonhole principle, $E = H$.  \hfill $\square$ \\
    \end{enumerate}

\end{enumerate}
\end{document}